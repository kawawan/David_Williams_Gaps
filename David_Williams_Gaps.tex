\documentclass[11pt, a4paper]{jsarticle}
    \usepackage{amsmath}
    \usepackage{amsthm}
    \usepackage[psamsfonts]{amssymb}
    \usepackage[dvipdfmx]{graphicx}
    \usepackage[dvipdfmx]{color}
    \usepackage{color}
    \usepackage{ascmac}
    \usepackage{amsfonts}
    \usepackage{mathrsfs}
    \usepackage{amssymb}
    \usepackage{graphicx}
    \usepackage{fancybox}
    \usepackage{enumerate}
    \usepackage{verbatim}
    \usepackage{subfigure}
    \usepackage{proof}


    %
    \theoremstyle{definition}
    %
    %%%%%%%%%%%%%%%%%%%%%%%%%%%%%%%%%%%%%%
    %ここにないパッケージを入れる人は,必ずここに記載すること.
    %
    %%%%%%%%%%%%%%%%%%%%%%%%%%%%%%%%%%%%%%
    %ここからはコード表です.
    %
   \newtheorem{axiom}{公理}[section]
    \newtheorem{defn}{定義}[section]
    \newtheorem{thm}{定理}[section]
    \newtheorem{prop}[thm]{命題}
    \newtheorem{lem}[thm]{補題}
    \newtheorem{cor}[thm]{系}
    \newtheorem{ex}{例}[section]
    \newtheorem{claim}{主張}[section]
    \newtheorem{property}{性質}[section]
    \newtheorem{attention}{注意}[section]
    \newtheorem{question}{問}[section]
    \newtheorem{prob}{問題}[section]
    \newtheorem{consideration}{考察}[section]
    \newtheorem{Alert}{警告}[section]
    \newtheorem{Rem}{注意}[section]
    %%%%%%%%%%%%%%%%%%%%%%%%%%%%%%%%%%%%%%
    %
    %定義や定理等に番号をつけたくない場合(例えば定理1.1等)は以下のコードを使ってください.
    %但し,例えば\Axiom*{}としてしまうと番号が付いてしまうので,必ず \begin{Axiom*} \end{Axiom*}の形で使ってください.
    \newtheorem*{axiom*}{公理}
    \newtheorem*{def*}{定義}
    \newtheorem*{thm*}{定理}
    \newtheorem*{prop*}{命題}
    \newtheorem*{lem*}{補題}
    \newtheorem*{ex*}{例}
    \newtheorem*{cor*}{系}
    \newtheorem*{claim*}{主張}
    \newtheorem*{property*}{性質}
    \newtheorem*{attention*}{注意}
    \newtheorem*{question*}{問}
    \newtheorem*{prob*}{問題}
    \newtheorem*{consideration*}{考察}
    \newtheorem*{alert*}{警告}
    \newtheorem*{rem*}{注意}
    \renewcommand{\proofname}{\bfseries 証明}
    %
    %%%%%%%%%%%%%%%%%%%%%%%%%%%%%%%%%%%%%%
    %英語で定義や定理を書きたい場合こっちのコードを使うこと.
    \newtheorem{Axiom+}{Axiom}
    \newtheorem{Definition+}{Definition}
    \newtheorem{Theorem+}{Theorem}
    \newtheorem{Proposition+}{Proposition}
    \newtheorem{Lemma+}{Lemma}
    \newtheorem{Example+}{Example}
    \newtheorem{Corollary+}{Corollary}
    \newtheorem{Claim+}{Claim}
    \newtheorem{Property+}{Property}
    \newtheorem{Attention+}{Attention}
    \newtheorem{Question+}{Question}
    \newtheorem{Problem+}{Problem}
    \newtheorem{Consideration+}{Consideration}
    \newtheorem{Alert+}{Alert}
    %
    %
    %%%%%%%%%%%%%%%%%%%%%%%%%%%%%%%%%%%%%%
    %数
    \newcommand{\NN}{{\mathbb{N}}} %自然数全体,
    \newcommand{\ZZ}{{\mathbb{Z}}} %整数環
    \newcommand{\QQ}{{\mathbb{Q}}} %有理数体
    \newcommand{\RR}{{\mathbb{R}}} %実数体
    \newcommand{\CC}{{\mathbb{C}}} %複素数体
    \title{Probability with Martingales のギャップとかメモ}
    \author{Twitter : @skbtkey}
    \date{}

\begin{document}
  \maketitle
  \begin{abstract}
    タイトル通り。ネットの海からこれを見つけ出した方は参考にしていただけると嬉しい。
    David Williams 著 "Probability with Martingales"。
  \end{abstract}

  \section{Chapter 6}
  \subsection{65ページ9行目}
    $c_1U_1 + c_2U_2 \sim c_1V_1 + c_2V_2$について。\\
    \begin{align*}
      &\{x \mid (c_1U_1 + c_2U_2)(x) \neq (c_1V_1 + c_2V_2)(x)\} \\
      \Leftrightarrow &\{x \mid (c_1U_1 + c_2U_2)(x) - (c_1V_1 + c_2V_2)(x) \neq 0\} =: A
    \end{align*}
    の測度が$0$であることを示せばよい。
    明らかに、
    \[
      c_1U_1 \sim c_1V_1, c_2U_2 \sim c_2V_2
    \]
    である。したがって、
    \begin{align*}
      A_1 := \{x \mid c_1U_1(x) \neq c_1V_1(x)\} \\
      A_2 := \{x \mid c_2U_2(x) \neq c_2V_2(x)\}
    \end{align*}
    の測度はそれぞれ$0$である
    \footnote{別に$A_1$などと名前を付けなくてもいいが(むしろ名前を付けない方がわかりやすい。)、紙面のスペースの都合上名前を付けている。}。
    このとき、次が成立。
    \[
      A \subset A_1 \cup A_2
    \]
    背理法で示す。上式の左辺から任意に$x$を取る。$x \notin A_1 \cup A_2$、つまり、
    \[
      x \in A_1^c \cap A_2^c
    \]
    と仮定する。このとき、$c_1U_1(x) = c_1V_1(x), c_2U_2(x) = c_2V_2(x)$であるから、
    \[
      (c_1U_1 + c_2U_2)(x) - (c_1V_1 + c_2V_2)(x) = 0\
    \]
    より、矛盾する。また、
    \begin{multline*}
      A = \{x \mid (c_1U_1 + c_2U_2)(x) - (c_1V_1 + c_2V_2)(x) < 0\} \\
      \cup \{x \mid (c_1U_1 + c_2U_2)(x) - (c_1V_1 + c_2V_2)(x) > 0\}
    \end{multline*}
    であるから、$A$は可測集合。したがって、測度の単調性より、$A$の測度は$0$。

  \subsection{65ページ10行目}
    $U_n \to U$より、
    $N \in \NN$が存在して、
    $n \ge N \Rightarrow ||U_n - U|| < \varepsilon \quad (\forall \varepsilon > 0)$
    である。
    $U_n \sim V_n, U \sim V$より、$||V_N - V|| < \varepsilon$が分かる。

  \subsection{67ページ10行目から12行目}
    (ii) $\Rightarrow$ (i)について。(逆向きは本の中で証明されています。)

    任意の$Z \in \mathcal{K}$を取ると、$||X - Z|| \ge ||X - Y|| + \alpha$となることを示すとよい。
    ここで、$\alpha > 0$である。
    しかし、$Z$として$Y + tZ \quad (t \in \RR)$とすればよい。
    これは$\mathcal{L}^p$の線形性のためである。
    なんとなれば、
    $Z = \frac{-Y+ W}{t} \quad (\forall W \in \mathcal{K})$とすればよい。
    $\langle X - Y, Z \rangle = \mathbf{E}[(X - Y)Z] = 0$
    即ち、$\mathbf{E}[XY] = \mathbf{E}[YZ]$が成り立つことに注意して、
    $||X - Y - tZ||$を計算すると分かる。






\end{document}
