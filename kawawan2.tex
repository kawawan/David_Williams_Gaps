% Chapter11から
\section{Chapter 11}
  ここは本当に素晴らしい章。
  しかし若干直観的に示している部分もあるのでそこだけ補足。
  \subsection{11.2}
    \subsubsection{108ページ式(D)について}
      直観的に明らかですが、きちっとした証明をすると少し煩雑でした。
      $\leq$の記号を用いているのは意図的で、$=$が成立するときがちゃんとあります。
      \begin{proof}
        次のように場合分けする。
        \begin{enumerate}
          \item 上向き横断がなかった場合
            \begin{enumerate}
              \item ゲームに参加しなかった場合
              \item ゲームに参加した場合
            \end{enumerate}
          \item 上向き横断があった場合
          \begin{enumerate}
            \item 最後の横断ののち、ゲームに参加しなかった場合
            \item 最後の横断ののち、ゲームに参加した場合
          \end{enumerate}
        \end{enumerate}

        [1.(a)について]
        1.は(a),(b)ともに、$U_N[a,b](\omega) = 0$のときである。
        1.(a)の時は、
        \[Y_N = 0\]
        であるので、
        \[Y_N \ge -(X_N - a)^-\]
        が成立。また、\underline{このときのみ等号が成立}する。

        [1.(b)について]
        ゲームに参加はしているので、どこかの時刻で$X$は$a$を下回っている。
        $m$をその最小の時刻とする。このとき、
        \[
          Y_N = X_N - X_m
        \]
        となるので、
        \[X_N - X_m \ge -(X_N - a)^-\]
        を示せばよい。
        \begin{align*}
          \text{(左辺)}-\text{(右辺)} &= X_N - X_m + (X_N - a)^-\\
          &= \begin{cases}
            X_N - X_m + 0 & (0 \ge -X_N + a)\\
            X_N - X_m - X_N + a & (-X_N + a > 0)
        \end{cases}\\
        &= \begin{cases}
          X_N - X_m  & (X_N \ge a) \\
          a - X_m  & (a > X_N)
        \end{cases}
        \end{align*}
        どちらの場合も、$>0$となる\footnote{$X_m$の定義!}。よって成立。

        [2.(a)の場合]
        $U_N[a,b](\omega) = k$とおく。これは2.(b)でも同様とする。
        上向き横断の定義より、ある時刻$t_k (\leq N)$が存在して、
        $X_{t_k} > b$である。
        横断を一回終えると所持金は少なくとも$(b-a)$は増えるので、
        \[Y_{t_k} > k(b-a) = (b-a) \times U_N[a,b](\omega)\]
        が成立する。
        \begin{align*}
          Y_N &= Y_{t_k} + (Y_N - Y_{t_k}) \\
          &> (b - a)U_N[a,b] + (Y_N - Y_{t_k})
        \end{align*}
        であることに注意すると、あとは
        \[Y_N - Y_{t_k} \ge -(X_N - a)^-\]
        を示せばよい。
        (a)の場合、つまり$t_k + 1$行こうゲームに参加しなかった場合、
        \[Y_N = Y_{t_k}\]
        だから、
        \[Y_N - Y_{t_k} = 0 \ge -(X_N - a)^-\]
        なので、成立する。

        [2.(b)の場合]
        \[Y_N - Y_{t_k} \ge -(X_N - a)^-\]
        を示ばよいというところまでは同じ。
        時刻$(t_k \leq)\, l \,(< N)$から再びゲームに参加したとする。
        このとき、
        \[
          Y_N - Y_{t_k} = Y_{t_k} + X_N - X_l - Y_{t_k} = X_N - X_l
        \]
        なので、
        \begin{equation*}
          X_N - X_l + (X_N - a)^- = \begin{cases}
            X_N - X_l & (X_N \ge a)\\
            a -X_l & (a> X_N)
        \end{cases}
        \end{equation*}
        どちらの場合も$>0$となる。
      \end{proof}

  \subsection{11.5}
    \subsubsection{109ページ12行目}
      \[
        \Lambda_{a,b} \subseteq \{\omega \mid U_{\infty}[a,b](\omega) = \infty\}
      \]
      について。

      \begin{proof}
        $\omega \in \Lambda_{ab}$をとる。
        $U_{\infty}[a,b](\omega) = k < \infty$と仮定する。
        $\Lambda_{ab}$の定義より、
        \[\liminf X_n(\omega) = \sup_{n}\inf_{\nu \ge n}X_{\nu}(\omega) < a\]
        この式から明らかなように
        \[\inf_{\nu \ge n}X_{\nu}(\omega) < a \quad (\forall n)\]
        同様に、
        \[\sup_{\nu \ge n}X_{\nu}(\omega) > b \quad (\forall n)\]
        $t_k$を最後の上向き横断時刻が達成された時刻とする。
        時刻$t_k + 1$について次が成立する。
        \[\inf_{\nu \ge t_k + 1}X_{\nu}(\omega) < a\]
        全ての$\nu \ge t_k + 1$について$X_{\nu} \ge a$とすると
        \footnote{$\min$でとっているならこの仮定はいらないがいまは$\inf$でとっているので必要}、
        $a$は数列$\{X_{\nu}(\omega)\}_{\nu = t_k +1}^{\infty}$の下界である。
        したがって、下限は下界の最大元であるから
        \[\inf_{\nu \ge t_k + 1}X_{\nu}(\omega) \ge a\]
        が従い、矛盾する。
        よって、ある$\xi_1 (\ge t_k + 1)$について
        \[X_{\xi_1} < a\]
        となる。
        同様の議論をすると、ある$\xi_2 (\ge \xi_1)$について
        \[X_{\xi_2} > b\]
        となる。これは、$k+1$回目の上向き横断が実現されたことになる。
        よって矛盾。よって$U_{\infty}[a,b](\omega) = \infty$
        が従う。
      \end{proof}
      \begin{rem*}
        $\limsup$と$\liminf$が一致していないということは、
        絵を描けば$X$のsample pathは上下に振動しまくっていることになる。
        だから直観的には明らかでしょう。
        もう少し短い証明があるかもですね。
      \end{rem*}

\section{Chapter 12}
  \subsection{12.1}
    \subsubsection{110ページ下から2行目}
      9.5節を参照せよとあります。参照しましょう。
      \begin{itembox}[l]{9.5の内容}
        $\mathcal{F} \supseteq \mathcal{G}$で、$\mathcal{G}$は
        $\mathcal{F}$のsub $\sigma$-alg.とする。
        $\mathcal{L}^2(\mathcal{F})$において
        $X$の$\mathcal{G}$に関する条件付き期待値の
        存在を言うとき、$X$に直交射影の定理(6.11参照)を用いて、
        とある確率変数$Y$の存在を言っていた。
        この時の$Y$は次を満たす。
        \begin{itemize}
          \item $(X - Y) \perp Z , \quad \forall Z \in \mathcal{L}^2(\mathcal{G})$
          \item $Y$は$\EE[X \mid \mathcal{G}]$の一変形
        \end{itemize}
      \end{itembox}
      さて、戻りましょう。
      教科書の110ページの下から3行目で
      \[
        \EE[M_v \mid \mathcal{F}_u] = M_u \quad (a.s)
      \]
      を示していた。$\mathcal{F}_v$は完備より、6.11の定理を使うと、
      ある$Y \in \mathcal{L}^2(\mathcal{F}_u)$が存在して
      \[(M_v - Y) \perp Z \quad (\forall Z \in \mathcal{L}^2(\mathcal{F}_u))\]
      となる。
      また、9.5よりこの$Y$は$\EE[M_v \mid \mathcal{F}_u]$の一変形だった。
      条件付き期待値はalmost surely で一意的なので、
      \[
        Y = M_u \quad (a.s)
      \]
      したがって、
      \[
        (M_v - M_u) \perp Z \quad (\forall Z \in \mathcal{L}^2(\mathcal{F}_u))
      \]
      となり、$(M-v -M_u)$は$\mathcal{L}^2(\mathcal{F}_u)$に直交する。
      $M_t, M_s$はともに$\mathcal{F}_u$可測で、
      $M$が$\mathcal{L}^2$有界より、
      \[
        M_t, \, M_s \in \mathcal{L}^2(\mathcal{F}_u)
      \]
      ゆえに、
      \[
        M_t - M_s \in \mathcal{L}^2(\mathcal{F}_u)
      \]
      したがって、とりわけ110ページの式(a)が成立するのである。

    \subsubsection{111ページTHEOREMの証明}
      直観的には明らかだけど...。

      \begin{proof}
        $(\Rightarrow)$
        $M$は$\mathcal{L}^2$有界とする。
        (b)式の右辺より、数列$\{\EE[M_n^2]\}_{n = 0}^{\infty}$
        は単調増加な数列で、仮定より、上に有界である。
        したがって、極限
        \[
          \lim_{n \to \infty}\EE[M_n^2]
        \]
        は有限の値に収束する。
        すなわち、
        \[
          \EE[M_0^2] + \sum_k \EE[(M_k - M_{k-1})^2)] < \infty
        \]
        したがって(c)の式が成立する。

        $(\Leftarrow)$
        (c)が成立するとする。
        \[
          \EE[M_n^2] \leq \EE[M_0^2] + \sum_k \EE[(M_k - M_{k-1})^2] < \infty
        \]
        である。
        $\EE[M_0^2] + \sum_k \EE[(M_k - M_{k-1})^2]$は
        $\{\EE[M_n^2]\}_{n = 0}^{\infty}$の上界である。
        よって、
        \[
          \sup_{n} \EE[M_n^2] \leq \EE[M_0^2] + \sum_k \EE[(M_k - M_{k-1})^2] < \infty
        \]
      \end{proof}

    \subsubsection{111ページ(c)式の2行下}
      $M_{\infty}$が$\mathcal{L}^2$に存在するのはFatou Lemma か言えたり。
      \begin{align*}
        \EE[M_{\infty}^2] = \EE[\liminf M_n^2] &\leq \liminf\EE[M_n^2]\\
        &\leq \sup_n \EE[M_n^2]\\
        &< \infty
      \end{align*}

    \subsubsection{111ページ一番下の行}
      (e)式が$=$でも成立するのは、
      \[
        \lim_{r \to \infty} \EE[(M_{n+r} - M_n)^2] = \EE[(M_{\infty} - M_n)^2]
      \]
      が成立するから。なおこの式が成立するのはノルムは連続関数だから。あ

  \subsection{112ページNotationの$A_n$の定義について}
    $A_n$は分散の和なので、実数列。確率変数列じゃないよ!
  \subsection{112ページ(*)の式の二つ目の$=$}
    分散を計算。
  \subsection{113ページ上から8行目}
    $N$がmaritingaleになることは計算しましょう。
    ここまでの議論で
    \[\sigma_k^2 = \EE[M_k^2 \mid \mathcal{F}_{k-1}] - M_{k-1}^2\]
    がalmost surelyで成り立つことが分かっている。
    この等式を用いて、
    \[\EE[N_m \mid \mathcal{F}_{m-1}] = N_{m-1}\]
    がalmost surelyで成り立つことを証明しよう。
  \subsection{113ページ上から12行目}
    $N^T$はmartingaleだからTower Propertyを用いれば、
    \[\EE[N_n^T \mid \mathcal{F}_0] = N_0 = 0\]
    が分かる。
  \subsection{113ページ(b)の証明の最後の三行}
    ここは不親切。"bounded"の意味を取り違えないように。
    翻訳すれば、
    「しかしながら、$\sum X_n$がa.s.で収束するから、任意の部分和$\sum_{k=1}^n X_k$
    も上に有界である。したがって、うまく実数$c$をとれば、$\PP(T= \infty) > 0$
    とすることができる。このことと、(**)の式により、
    $A_{\infty} := \sum \sigma_k^2 < \infty$であることが分かる。」
    のようになる。
    (この部分の日本語訳の本を見てみましたが、あまり訳がうまくないとおもいます。
    逐語訳を意識しすぎているのと、学術的な言いまわしに凝りすぎていて日本語として意味が
    掴みにくいです。)

    さて、「部分和が上に有界」とは、次のような意味です。
    \begin{center}
      $\omega$を一つ固定するごとに、$n$に関する数列
      $\{\sum_{k=1}^n X_k(\omega)\}$は上に有界である。
    \end{center}
    次のような意味ではないことに気を付けてください\footnote{このような誤解を生んでしまうような書き方は避けるべきだと思います。
    丁寧に「$\omega$を任意に一つ固定すれば」という文言を付け足せば、
    それだけで悲しい誤解を生まなくて済むのに。}。
    \begin{center}
      関数列$\{\sum_{k=1}^n X_k\}_{n \in \NN}$は一様有界である。すなわち、
      \[\exists K ,\forall \omega, \forall n, \quad |\sum_{k=1}^n X_k(\omega)| \leq K\]
    \end{center}

    \subsubsection{$\PP(T= \infty) > 0$について}
      $Y$を次のように置きます。
      \[Y(\omega) := \sup_n \left|\sum_{k=1}^n X_k(\omega) \right|\]
      先ほど言ったように、$Y(\omega) < \infty$です。
      よって、$Y(\omega) < c$となるような実数$c$があります。
      \[\PP(Y < c) \rightarrow 1 \quad (c \rightarrow \infty)\]
      測度の単調性から$c$を大きくしていけばいつしか$\PP(Y < c)$は$0$を超えます。
      だからそのような$c$をとれば、$T_c := \inf\{r \mid |M_r| > c\}$について
      \[\PP(T_c = \infty) > 0\]
      となります。

    \subsubsection{証明の完成}
      さきほどとった$\PP(T_c = \infty) > 0$となるような$c$について次のような式が成り立ちます。
      \begin{align*}
        \EE[A_{T \wedge n}] \leq& (K + c)^2 \\
        \EE[\sum_{k = 1}^{T_c \wedge n}\Var(X_k)] =& \left( \sum_{k=1}^n \Var(X_k)\right) \PP(T_c > n)\\
        &+ \left( \sum_{k=1}^{T_c} \Var(X_k)\right) \PP(T_c \leq n)\\
        (K + c)^2 \ge& \left( \sum_{k=1}^n \Var(X_k)\right) \PP(T_c > n)\\
        (K + c)^2 \ge& \left( \sum_{k=1}^{\infty} \Var(X_k)\right) \PP(T_c = \infty)\\
        \sum_{k=1}^{\infty} \Var(X_k) \leq& \frac{(K + c)^2}{\PP(T_c = \infty)}
      \end{align*}

  \subsection{114ページ2行目}
    "only if"の方、つまり
    \[\sum \varepsilon_n a_n \quad \text{converges(a.s.)} \Rightarrow \sum a_n^2 < \infty\]
    はここの文章に書かれている以上の条件がいります。
    それは、$\{\varepsilon_n a_n\}_{n \in \NN}$が一様有界であることです。
    つまり数列$\{a_n\}_{n \in \NN}$が有界であるという条件が必要です。

  \subsection{114ページ4行目}
    もう8章も前のことなので、影が薄いですが、「確率変数列の和が各点収束する」
    という事象は末尾$\sigma$-alg.の事象です。
    つまり、この事象が起こる確率はコルモゴロフの0-1法則より、
    1か0です。このことに注意しましょう。

  \subsection{115ページ下から14行目}
    (12.2.a)を適用しているのは$(X_n - \EE[X_n])$に使っています。
    使うからにはこれが独立確率変数の列であることと、
    各$n$について平均が0で分散が有限であることを確かめることを忘れずに。

  \subsection{116ページ証明の5行目}
    "... for all but finitely many $n$"という文章は直観的でパット見分かりやすいですが、
    数式で書いた方がよいと思います。そうでないと扱い方が分かりません。
    (BC1)で従うのは、
    \[
      \PP(\limsup \{|X_n| > K\}) = 0
    \]
    であること。$\limsup\{\text{事象}F\}$は、「事象$F$が無限回起こること」
    を意味している。
    これは、「有限個の$n$を除いて事象$F$が起こる」ことではない\footnote{実質無限回起きてるけど}。
    文学的な表現だけを用いて証明を進めると
    このように意味が分からなくなってしまうので、
    数式を書くべきところではちゃんとするべきだと思いました。
    $i.o.$や$e.v.$という用語も直観的に議論を進めるためには大いに重要だと思います。
    しかし、数式に戻してちゃんと議論をしたいときには
    その数式を証明に明記しておかないと分かりにくいと思います。
    こういうことが多いからこの本嫌いなんだよ。
    \footnote{"for all but finitely mayn $n$"という表現もここが初出じゃないの?BC1を証明した時点で$i.o.$や$e.v.$という用語を導入しているのだから、そっちを使ってほしい。}

    というわけで、どのようにして証明5行目の式が得られたかを書きます。
    (BC1)より、$\PP(\limsup\{|X_n| > K\}) = 0$が得られた。
    この集合の補集合を考える。
    \begin{align*}
      1 &= \PP((\limsup\{|X_n| > K\})^c) \\
      &= \PP(\liminf\{|X_n| \leq K\})) \\
      &= \PP(\liminf\{X_n^K = X_n\})\\
      &= \PP(X_n^K \neq X_n \quad e.v.)\\
      &= \PP\left(\bigcup_m \bigcap_{n \ge m}\{X_n^K = X_n\}\right)
    \end{align*}

  \subsection{116ページ証明の6行目}
    "It is therefore clear that we need only show that ...;"のところ。
    $\omega$を固定するごとに有限個の$n$しか$X_n(\omega) \neq X_n^K(\omega)$
    とならない。だから直観的には$\sum X_n^K$がalmost surely で収束することを
    示せば良さそうです。じつは、
    \[
      \sum X_n \text{が収束}(a.s) \Leftrightarrow \sum X_n^K \text{が収束}(a.s.)
    \]
    です。
    これを示すためのちょっとした補題。
    \begin{lem*}
      $\PP$を確立測度とする。$A,B$は確率$1$の事象とする。
      このとき、$\PP(A \cap B) = 1$である。
    \end{lem*}
    \begin{proof}
      $\PP((A \cap B)^c) = \PP(A^c \cup B^c) = 0$であることが仮定よりわかる。
    \end{proof}
    先ほどの同値命題の証明です。
    \begin{proof}
      \begin{gather*}
        A = \bigcup_m \bigcap_{n \ge m}\{X_n^K = X_n\} \\
        B = \{\sum X_n \text{が収束}\} \\
        C =\{\sum X_n^K \text{が収束}\}
      \end{gather*}
      とおく。

      ($\Rightarrow$)

      $\omega \in A \cap B$をとる
      \footnote{$A \cap B \neq \emptyset$は確認してください}。
      このとき、ある$n_{\omega}$が存在して、すべての$n \ge n_{\omega}$
      にたいして、$X_n^K(\omega) = X_n (\omega)$になる。
      このとき、
      \begin{align*}
        \sum X_n^K(\omega) &= \sum_{k=1}^{n_{\omega}-1} X_k^K(\omega) + \sum_{k\ge n_{\omega}} X_k^K(\omega) \\
        &=\sum_{k=1}^{n_{\omega}-1} X_k^K(\omega) + \sum_{k\ge n_{\omega}} X_k(\omega)
      \end{align*}
      また、
      \[
        \sum X_n(\omega) = \sum_{k=1}^{n_{\omega}-1} X_k(\omega) + \sum_{k\ge n_{\omega}} X_k(\omega)
      \]
      であり、$\omega \in B$であることに気を付ければ$\omega \in C$であることがわかる
      \footnote{ここちょっと面倒くさくなった(おなかすいててん)}。
      よって、$A \cap B \subset C$が分かる。
      よって先ほどの補題より、$\PP(C) = 1$であることが分かる。

      同様にすれば$(\Leftarrow)$もわかる。
    \end{proof}

  \subsection{116ページ'only if'パートの証明}
    最初の三行から"$|X_n|> K$ for only finitely many $n$"が分かります。
    ここまではいいでしょう。ここからどのようにして(BC2)を用いたのかを書きます。
    有限個の$n$を除いて$|X_n|> K$ということは、ある番号$m$からすべて
    $|X_n| \leq K$となっていることを表します。
    すなわち、
    \[
      \PP(\liminf \{|X_n| \leq K\}) = 1
    \]
    です。
    これはつまり
    \[
      \PP(|X_n| > K \quad i.o.) = 0
    \]
    を表します。
    $\sum \PP(|X_n| > K ) = \infty$を仮定します。$X_n$は独立だから、
    事象$\{|X_n| > K\}$は独立な事象。
    したがって、(BC2)を用いることができて、
    \[
      \PP(|X_n| > K \quad i.o.) = 1
    \]
    が従います。しかしこれは矛盾。よって、
    \[
      \sum \PP(|X_n| > K ) < \infty
    \]
    となります。\footnote{$\PP(|X_n| > K \quad i.o.) = 0$がわかっているのだから、
    その和である$\sum \PP(|X_n| > K ) < \infty$は自明なのでは?
    という疑問がありますが、これを自明というのは早計です。
    $i.o.$の二文字が付いてなければ確かにそうですが、
    今回はついているのでBC2を挟まなければならないというお話です。}
    これにより、(i)が証明できました。
    あとは前半の証明と同様です。

  \subsection{117ページ3行目}
    収束することの定義を使ってこの不等式が得られています。
    実際は、
    \[
      |v_k - v_{\infty}| < \varepsilon
    \]
    を変形して得られた式です。

  \subsection{118ページ下から4行目}
    $\RR$上の$\RR$値関数$h_n$を次のように定める。
    \[
      h_n(x) := \begin{cases}
        x & (|x| \leq n) \\
        0 & (|x| > n)
    \end{cases}
    \]
    すると、$Y_n,Z_n$はそれぞれ$h \circ X_n, h \circ X$とかけます。
    よって、$X,X_n$の分布を$\Lambda, \Lambda_n$と書くことにすれば
    \footnote{同じものだけど区別のために記号を導入しました}、
    \begin{align*}
      \EE[Z_n] &= \EE[h \circ X] \\
      &= \int_{\RR} h(x) \Lambda_X(dx) \\
      &= \int_{\RR} h(x) \Lambda_{X_n}(dx) \\
      &= \EE[h \circ X_n]\\
      &= \EE[Y_n]
    \end{align*}
    となる。

  \subsection{119ページ上から2,3行目について}
    \subsubsection{二つ目の$=$}
      $X_n,X$は同分布より。
    \subsubsection{三つ目の$=$}\footnote{119ページ上から3行目の最初の$=$}
      \begin{align*}
        \sum \PP(|X| > n) &= \sum \EE[I_{\{|X| > n\}}] \\
        &= \lim_n \EE[\sum_{k =1}^n I_{\{|X| > n\}}] \\
        &= \EE[\lim_n \sum_{k =1}^n I_{\{|X| > n\}}] \\
        &= \EE[\sum_{n =1}^{\infty} I_{\{|X| > n\}}]
      \end{align*}
      ここで、三つ目の$=$については(MON)を用いた。
    \subsubsection{四つ目の$=$}
      右辺の
      \[
        \sum_{1 \leq n < |X|}1
      \]
      は、
      \[
        \sum_{m \in \{n \in \NN \, \mid \, 1 \leq n < |X|\}}1
      \]
      ということ。
      $\sum$の下に不等号で$a<n<b$のように条件が書かれているときは、
      だいたい$a,b$は整数だったりすることが多いもんね...。わからなくなるよ...。

      ここの期待値が等しいのは、関数として
      \[
        \sum_{n =1}^{\infty} I_{\{|X| > n\}} = \sum_{1 \leq n < |X|}1
      \]
      が成立しているから。$\omega$を任意に固定して
      \[
        \sum_{n =1}^{\infty} I_{\{|X(\omega)| > n\}} = \sum_{1 \leq n < |X(\omega)|}1
      \]
      であることを確認してみよう。

      別の資料を調べていたらtruncationである$Y_n$の不等号の位置を逆にしたもの、
      つまり、
      \[
        Y_n := \begin{cases}
          X_n & (|X_n| < n)\\
          0 & (|X_n| \ge n)
      \end{cases}
      \]
      と定義してKolmogorov's truncation Lemmaを証明しているものを見つけました
      \footnote{$Z_n$も同様に定義します。}。
      この場合は$\displaystyle{\sum_{1 \leq n < |X|}}$と書かずにガウス記号
      \footnote{このpdfでは床関数を使って書いています}を使って
      $\displaystyle{\sum_{n = 1}^{\lfloor |X| \rfloor}}$
      と慣れた形で計算できます。

  \subsection{119ページ(iii)の証明の二行目の不等式}
    \begin{align*}
      &\sum \frac{\Var(Y_n)}{n^2} \leq \sum \frac{\EE[Y_n^2]}{n^2} = \sum \frac{\EE[|X|^2 I_{\{|X| \leq n\}}]}{n^2} \\
      =& \EE\left[|X|^2 \sum \frac{1}{n^2}I_{\{|X| \leq n\}}\right] = \EE\left[|X|^2 \sum_{n \ge \max(1,|X|)} \frac{1}{n^2}\right] \\
      \leq& \EE\left[|X|^2 \frac{2}{\max(1,|X|)}\right] \leq 2\EE[|X|] < \infty
    \end{align*}
    二つ目の$=$はFubini、三つ目の$=$は$\EE$の中身が関数として一致していることから言える。
    教科書のように$f$を定義すれば綺麗に書くことができますが、
    理解のためにはこのようにしてゴリゴリに書いた方が良いと思います。

  \subsection{119ページ下から二行目}
    "we need onky show that..."のところ。
    $Y_n,X_n$は有限個しか違わないので、$n \to \infty$とすればどうせ異なる部分は
    $0$に収束して消えちゃうからね。

  \subsection{120ページDoob decompositionについて}
    一意性の証明どこ......ここ......?

    doob分解のa.s.一意性の証明を書きます。
    \begin{proof}
      $X$が$X_0 + A_n + M_n, \,\, X_0 + \tilde{A}_n + \tilde{M}_n$と二通りに分解できたとする。
      これにより、
      \[A_n + M_n = \tilde{A}_n + \tilde{M}_n\]
      が分かる。すべて左辺に移行して
      \[A_n + M_n - \tilde{A}_n - \tilde{M}_n = 0\]
      両辺$\mathcal{F}_{n-1}$で条件付き期待値を取って、
      \[\EE[A_n + M_n - \tilde{A}_n - \tilde{M}_n \mid \mathcal{F}_{n-1}] = 0\]
      すると、martingaleやprevisibleであることを使って、
      \[A_n + M_{n-1} - \tilde{A}_n - \tilde{M}_{n-1} = 0\]
      ゆえに、
      \[A_n = - M_{n-1} + \tilde{A}_n + \tilde{M}_{n-1}\]
      である。この$A_n$をいい感じのところに代入すると,
      \[M_n  - \tilde{M}_n = M_0  - \tilde{M}_0 = 0 \]
      よって、
      \[A_n  = \tilde{A}_n\]
      も分かる。なお、ここでの$=$にはすべてa.s.が付くが、省略している。
    \end{proof}

    \subsubsection{(b)の証明}
      \begin{align*}
        \EE[X_n \mid \mathcal{F}_{n-1}] &\ge X_{n-1}\\
        X_0 + M_{n-1} + A_n &\ge X_0 + M_{n-1} + A_{n-1}\\
        A_n &\ge A_{n-1}
      \end{align*}
      上から下、下から上どちらの式変形もできるから証明終了。

  \subsection{122ページ(c)の証明}
    \[\infty > \sup_n \EE[M_n^2] = \sup_n \EE[A_n] = \lim_n \EE[A_n] = \EE[A_{\infty}]\,\,\underline{< \infty}\]
    右から見ても、左から見てもいい感じ!

  \subsection{122ページ(d)の証明}
    $\EE[M_n^2 - M_{n-1}^2 \mid \mathcal{F}_{n-1}]$の$M_n^2,M_{n-1}^2$をdoob分解の形に直して計算すると、
    \[\EE[M_n^2 - M_{n-1}^2 \mid \mathcal{F}_{n-1}]  = A_n - A_{n-1}\]
    が分かる。一方、
    \begin{align*}
      \EE[M_n^2 - M_{n-1}^2 \mid \mathcal{F}_{n-1}] &= \EE[M_n^2 + M_{n-1}^2 - 2M_{n-1}^2 \mid \mathcal{F}_{n-1}]\\
      &= \EE[M_n^2 + M_{n-1}^2 \mid \mathcal{F}_{n-1}] - 2M_{n-1}^2 \\
      &= \EE[M_n^2 + M_{n-1}^2 \mid \mathcal{F}_{n-1}] - 2M_{n-1}\EE[M_{n} \mid \mathcal{F}_{n-1}] \\
      &= \EE[M_n^2 + M_{n-1}^2 \mid \mathcal{F}_{n-1}] - 2\EE[M_{n-1}M_{n} \mid \mathcal{F}_{n-1}] \\
      &= \EE[M_n^2 + M_{n-1}^2 - 2M_{n-1}M_{n} \mid \mathcal{F}_{n-1}]\\
      &= \EE[(M_n - M_{n-1})^2 \mid \mathcal{F}_{n-1}]
    \end{align*}

  \subsection{123ページ上から3行目}
    \[B = \left( \bigcup_k \{S(k) = \infty\} \right) \cap \{\lim M_n \text{が存在しない}\}\]
    とする。すると、
    \[B \subset \{\lim M_{n \wedge S(k)} \text{が存在しない}\}\]
    が分かるので\footnote{確かめてね}、(c)より
    \[\PP(B) = 0\]
    が分かる。

  \subsection{123ページ上から14行目}
    (e)と(f)から矛盾が導き出せるらしい。
    \footnote{教科書の$T(c)$という書き方も気に入らないですねえ。
    stopping timeは確率変数だからカッコの中には$\omega$を書きたくなる。
    $T_c$とかにしてほしいよね。}

    $A$はincreasing process で、$A_0 = 0$より、
    \begin{align*}
      A_{T(c) \wedge n}I_{\{T(c) = \infty\}} &\leq A_{T(c) \wedge n}\quad (a.s.)\\
      A_{n}I_{\{T(c) = \infty\}} &\leq A_{T(c) \wedge n}\quad (a.s.)\\
      \EE[A_{n}I_{\{T(c) = \infty\}}] &\leq \EE[A_{T(c) \wedge n}]\\
      \EE[\lim_{n \to \infty}(A_{n}I_{\{T(c) = \infty\}})] &\leq (c + K)^2 \quad \text{[(MON) and (f)]}\\
      \EE[A_{\infty}I_{\{T(c) = \infty\}}] &\leq (c + K)^2
    \end{align*}
    したがって、
    \[\PP(\{T(c) = \infty\} \cap \{A_{\infty} = \infty\}) = 0\]
    が分かる。これは(e)と矛盾する。

  \subsection{123ページ下から5行目}
    "Since $(1 + A)^{-1}$ is a bounded previsible process"の"bounded"は
    "almost surely bounded"の意味でとりましょう。
    ここも正しいことを言っているのに言い方が雑なせいで誤解を生んでしまったところですね。

    さて、$W$がmartingaleであることは、10.7の定理を使っています。
    しかし、10.7で出てきたprevisible processは「非負、一様有界」という条件が付いています。
    特に「一様有界」の条件はmartingaleの可積分性の証明に用います。
    しかし、現在の12.14の$(1 + A)^{-1}$というprevisible processは「一様有界」という条件までは
    課していません。
    なので、12.14の$W$がmartingaleであることを示すときに、$\EE[|W|] < \infty$
    であることは個別に証明をした方が良いと思います。

  \subsection{124ページ上から3行目}
    $(X_n)$をi.i.d.r.v.sとし、各$X_n \in \mathcal{L}^2$とおきます。
    $M_n = X_n - \EE[X_n]$とおけば、$(M_n)_{n \ge 0}$は$\mathcal{L}^2$-martingaleになります。
    この$M$に対して、$\dfrac{M_n}{A_n}$に(a)を適応すれば、確かに強法則を見れます。

  \subsection{124ページ下から7行目}
    "Then (you check!)"と書かれているので確かめるぞい。
    \begin{align*}
      A_n &= \sum_{k=1}^n \EE\left[(M_k - M_{k-1}^2) \mid \mathcal{F}_{k-1}\right] \\
      &= \sum_{k=1}^n \EE[(Z_k - Z_{k-1} - (Y_k - Y_{k-1}))^2 \mid \mathcal{F}_{k-1}]\\
      &= \sum_{k=1}^n \EE[(I_{E_k} - (\xi_k))^2 \mid \mathcal{F}_{k-1}]\\
      &= \sum_{k=1}^n \EE[I_{E_k} - 2I_{E_k}\xi_k  + \xi_k^2 \mid \mathcal{F}_{k-1}]\\
      &= \sum_{k=1}^n \Bigl(\EE[I_{E_k} \mid \mathcal{F}_{k-1}] - \xi_k\EE[2I_{E_k} \mid \mathcal{F}_{k-1}]  + \EE[(\xi_k)^2 \mid \mathcal{F}_{k-1}] \Bigr)\\
      &= \sum_{k=1}^n (\xi_k -2\xi_k\xi_k + \xi_k^2) \\
      &= \sum_{k=1}^n \xi_k(1 - \xi_k) \\
      &= \sum_{k=1}^n (\xi_k - \xi_k^2) \\
      &= Y_n - \sum_{k=1}^n \bigl( \PP(E_n \mid \mathcal{F}_{k-1}) \bigr)^2 \leq Y_n - 0 = Y_n
    \end{align*}

\section{Chapter 13}
  この章あんまり書くことない。
  \subsection{126ページの補題の主張について}
    論理式で書いたらこんな感じ。
    \[\forall X \in \mathcal{L}^1 \Rightarrow \bigr(\forall \varepsilon > 0, \exists \delta > 0 \quad s.t.\quad \forall F \in \mathcal{F} , \PP(F) < \delta \Rightarrow \EE[|X| ; F] < \varepsilon \bigl)\]

  \subsection{129ページ下から3行目}
    $X_n \rightarrow X \quad (a.s.)$ならば$\PP(|X_n - X|> \varepsilon \quad i.o.) = 0$
    なのか?
    $X_n \rightarrow X \quad (a.s.)$なので、
    \[
      \PP \Bigr(\{\omega \mid \forall \varepsilon > 0, \exists N \in \NN, \forall n \ge N, |X_n(\omega) - X(\omega)| < \varepsilon\}\Bigl) = 1
    \]
    である。表記の簡略化のために
    \[F := \{\omega \mid \forall \varepsilon > 0, \exists N \in \NN, \forall n \ge N, |X_n(\omega) - X(\omega)| < \varepsilon\}\]
    とする。
    さて、今$\varepsilon^{\prime} > 0$を一つ固定する。$\omega \in F$をとると、
    定義からある$N_{\varepsilon^{\prime}} \in \NN$が存在して、任意の$n \ge N_{\varepsilon^{\prime}}$
    について
    \[|X_n(\omega) - X(\omega)| < \varepsilon^{\prime}\]
    であるから、
    \[\omega \in \{\omega \mid \exists N \in \NN, \forall n \ge N, |X_n(\omega) - X(\omega)| \leq \varepsilon^{\prime}\} =: G\]
    したがって、
    \[F \subset G\]
    である。$G$の定義に注目すると、
    \[G = \bigcup_n \bigcap_{m \ge n} \Bigl\{ |X_m - X| \leq \varepsilon^{\prime} \Bigr\} = \liminf_{n \to \infty}\Bigl\{ |X_m - X| \leq \varepsilon^{\prime} \Bigr\} \]
    よって、
    \[1 = \PP(F) \leq \PP(G) = \PP(|X_m - X| \leq \varepsilon^{\prime} \quad e.v.)\]
    以上により、
    \[\PP(|X_m - X| > \varepsilon^{\prime} \quad i.o.) = 0\]

    \subsection{131ページ下から11行目}
      "we can choose $K$ so that ..."とある。
      が、UIであることを用いたときの$K$と13.1.(b)を用いたときの$K$は異なるかもしれない。
      よって、前者の$K$を$K_{UI}$、後者の$K$を$K_{(b)}$と書くことにすれば\footnote{添え字のセンスがゴミだな自分}、
      \[K = \max\{K_{UI}, K_{(b)}\}\]
      とすればよい。

    \subsection{132ページ上の3行}
      前節の$K$に関する注意と同様の注意。
      "we can choose $\delta > 0$ such that whenever ..."とあるが、
      今回も$X_n$および、$X$ごとに$\delta$の値が変わってくる。
      全部で$N+1$個とれる$\delta$のうち最小のものを選べばよい。

    \subsection{$\mathcal{L}^p$(-bounded) $[p \ge 1]$とUIの関係}
      図に書いたよ。

\section{Chapter 14}
  この章は少し大変かもしれない。
  \subsection{135ページ上から3行目}
    $M_{\infty}$が$\mathcal{F}_{\infty}$なことについて。
    $M_n \in \mathcal{F}_{\infty} \, (\forall n)$だから。

  \subsection{135ページ上から7行目}
    "However, for each $n$, $\eta$ is $\mathcal{T}_n$ measurable,
     and hence (see Remark below) is independent of $\mathcal{F}_n$."
    のところ。
    下の"Remark"によくわからない文書が書かれていますが、要するに14.3の証明でも
    4章の証明で示したことを利用してますよってこと。
    それがこの部分。

    $\eta$\footnote{ここはわざわざ$\eta$を導入するよりも、$I_F$のまま計算を進めたほうが絶対に分かりやすい。}
    は$\mathcal{T}_n$可測より、$\sigma(\eta) \subset \mathcal{T}_n$になる。
    よって、$\eta$と$\mathcal{F}_n$が独立。

  \subsection{136ページ:14.4の証明について}
    後ろ向きのマルチンゲールは見慣れないですが、定義通りに計算すれば大丈夫。
    \subsubsection{$\{\mathcal{G}_{-n}\}$はfiltrationになっているか}
      filtrationは、
      \[\mathcal{G}_{k} \subset \mathcal{G}_{k+1} \,\, (\forall k)\]
      が成立していればいい。$n >0$とすると、
      \[\mathcal{G}_{-n} \subset \mathcal{G}_{-n+1}\]
      は仮定より成立している。

    \subsubsection{マルチンゲールの性質は満たしているか}
      $n > 0$とすると、Tower Propertyを用いれば
      \[\EE[M_{-n} \mid \mathcal{G}_{-n - 1}] = M_{-n - 1}\]
      と計算できるのでマルチンゲール。

    \subsubsection{$\mathcal{L}^1$-boundedであるか}
      $n>0$を任意に一つ固定する。$M_{-n}$の定義より、
      \[\EE[|M_{-n}|] = \EE[|\EE[\gamma \mid \mathcal{G}_{-n}]|] \leq \EE[\EE[|\gamma| \mid \mathcal{G}_{-n}]] = \EE[|\gamma|] < \infty\]
      よって、
      \[\{M_{-n} \mid n > 0\}\]は$\mathcal{L}^1$-bounded martingaleになる。

    \subsubsection{残り}
      $-r \leq -n < 0$とする。$G \in \mathcal{G}_{\infty} \subset \mathcal{G}_{-r} \subset \mathcal{G}_{-n}$とする。
      あとはおなじだ~がんばれ~。

  \subsection{137ページ上から7行目}
    \begin{align*}
      L = \limsup_n \frac{S_n}{n} &= \limsup_n \frac{S_{n+k}}{(n+k)} \\
      &= \limsup_n \left(\frac{X_1 + \cdots + X_k}{n+k} + \frac{X_{k+1} + \cdots + X_{k+n}}{k+n}\right)\\
      &= \limsup_n \left(\frac{X_{k+1} + \cdots + X_{k+n}}{n} \times \frac{n}{n+k}\right)\\
      &= \limsup_n \left(\frac{X_{k+1} + \cdots + X_{k+n}}{n} \right)
    \end{align*}

  \subsection{137ページExercise}
    Scheffe使えるのかこれ......?それの証明は分からなかったけど、別解をご用意しました。

    $X_1 \in \mathcal{L}^1$より、$\{\EE[X_1 \mid \mathcal{G}_{-1}] ; n \ge 1\}$
    はUI。
    \[\frac{S_n}{n} \rightarrow \mu \,\,(a.s.)\]
    より、
    \[\frac{S_n}{n} \rightarrow \mu \,\,(\text{in prob.})\]
     したがって、
     \[\frac{S_n}{n} \rightarrow \mu \,\,(\text{in }\mathcal{L}^1)\]

  \subsection{139ページ下から8行目}
    \[
      \EE[e^{\theta S_n}] = e^{\frac{1}{2}\theta^2 n}
    \]
    であることがはよく知られていますが、
    この本ではまともに標準正規分布について扱っていません。
    なのでちゃんとガウス積分をして確かめるべきです。

  \subsection{140ページ上から1行目}
    $\theta = \frac{c}{n}$と置いた理由。139ページの最後の行で、
    \[
      e^{-\theta c}e^{\frac{1}{2}\theta^2n}
    \]
    が得られている。$e$の指数を$\theta$に関して平方完成すると、
    $\theta = \frac{c}{n}$とした場合に、最小になることが分かる。

  \subsection{140ページ上から8行目}
    \[
      \sup_{k \leq K^n}S_k \leq c_n
    \]
    が a.s. で成立すると書いてあるが、地味に注意が必要。
    BC1で得られるのは
    \[
      \PP\Bigl(\Bigl\{\sup_{k \leq K^n}S_k < c_n \Bigr\} \quad e.v\Bigl) = 1
    \]
    といこと。ここで、
    \[
      \Bigl\{\sup_{k \leq K^n}S_k < c_n \Bigr\} \subset \Bigl\{\sup_{k \leq K^n}S_k \leq c_n \Bigr\}
    \]
    となっている\footnote{多くの場合この二つの集合は等しくなりますが......}ので、測度の単調性より、
    \[
      1 = \PP\Bigl(\Bigl\{\sup_{k \leq K^n}S_k < c_n \Bigr\} \quad e.v\Bigl) \leq \PP\Bigl(\Bigl\{\sup_{k \leq K^n}S_k \leq c_n \Bigr\} \quad e.v\Bigl)
    \]

  \subsection{140ページ上から5行目}
    14.8(b)を使っているが、使うための条件は満たしているのだろうか。
    $F_n$の中身の式を変形すると、
    \begin{align*}
      S_{N^{n+1}} - S_{N^n} &> (1 - \varepsilon)\sqrt{2}\sqrt{N^{n+1} - N^n}\sqrt{\log\log(N^{n+1} - N^n)} \\
      \frac{S_{N^{n+1}} - S_{N^n}}{\sqrt{N^{n+1} - N^n}} &> (1 - \varepsilon)\sqrt{2}\sqrt{\log\log(N^{n+1} - N^n)}
    \end{align*}
    とできる。ここで、各$n$に対して$X_n \sim \mathcal{N}(0,1)$だから、
    $(S_{N^{n+1}} - S_{N^n}) \sim \mathcal{N}(0,N^{n+1} - N^n)$であり、
    \[
      \frac{S_{N^{n+1}} - S_{N^n}}{\sqrt{N^{n+1} - N^n}} \sim \mathcal{N}(0,1)
    \]
    である。よって、14.8(b)を適用することができる。

  \subsection{140ページ下から2行目}
    感覚的にはそんな感じがしますが、
    \[
      \frac{1}{\sqrt{2\pi}}\left( y + \frac{1}{y} \right)^{-1}\exp\left( -\frac{y^2}{2}\right)
    \]
    をちゃんと不等式評価しましょう。

    \subsubsection{$\exp (-\frac{y^2}{2})$について}
      \begin{align*}
        \exp\left(-\frac{y^2}{2}\right) &= \exp\left(-(1-\varepsilon)^2 2\log\log(N^{n+1} - N^n)\times\frac{1}{2}\right)\\
        &= \frac{1}{\exp\bigl((1-\varepsilon)^2\log\log (N^{n+1} - N^n)\bigr)}\\
        &\ge \frac{1}{\exp\bigl((1-\varepsilon)^2\log\log N^{n+1}\bigr)}\\
        &= \frac{1}{\bigl(\exp(\log\log N^{n+1})\bigr)^{(1-\varepsilon)^2}}\\
        &= \frac{1}{((n+1)\log N)^{(1-\varepsilon)^2}}\\
        &\ge \frac{1}{(2n\log N)^{(1-\varepsilon)^2}}\\
        &= \frac{1}{(2\log N)^{(1-\varepsilon)^2}} \times \frac{1}{n^{(1-\varepsilon)^2}}\\
        &\ge \frac{1}{(2\log N)^{(1-\varepsilon)^2}} \times \frac{1}{n}\\
      \end{align*}

      なお、この式のうち、下から三行目の$\ge$は
      \[(n+1) \log N \leq 2n\log N\]
      が$n\ge1$で成り立つことを用いている。実際、$n\ge1$において、
      \begin{align*}
        2n\log N - (n+1) \log N &= n\log N - \log N\\
        &= (n-1)\log N \\
        &\ge 0
      \end{align*}
      である。

    \subsubsection{$\left( y + \frac{1}{y} \right)^{-1}$について}
      $x \in \RR$とするとき、もし$x \ge 3$ならば、
      \[\frac{x}{x^2 + 1} \ge \frac{1}{2x}\]
      が成り立つ。実際、$x\ge3$ならば、
      \[\frac{2x^2}{x^2 + 1} \ge 1\]
      であることが分かるから。

      $y\ge 3$のときを考えてこの評価を用いたい。$y \ge 3$とはすなわち、
      \begin{align*}
        (1 - \varepsilon)\sqrt{2\log\log(N^{n+1} - N^n)} &\ge 3\\
        \log\log N^n(N-1) &\ge \frac{9}{2(1- \varepsilon)^2}\\
        \log\Bigl(n\log N + \log(N-1) \Bigr) &\ge \frac{9}{2(1- \varepsilon)^2}\\
        n\log N +\log(N-1) &\ge \exp\left(\frac{9}{2(1- \varepsilon)^2}\right) \\
        n &\ge \frac{1}{\log N} \exp\left(\frac{9}{2(1- \varepsilon)^2}\right) - \frac{\log(N-1)}{\log N}
      \end{align*}
      すなわち、
      \[n \ge \left[ \frac{1}{\log N} \exp\left(\frac{9}{2(1- \varepsilon)^2}\right) \right] =\colon \alpha(>0)\]
      のときは$y \ge 3$が成立する。ただし、ここで$[\quad]$はガウス記号を表している。

      $y \ge 3$のとき、つまり$n \ge \alpha$のとき、
      \begin{align*}
        \left( y + \frac{1}{y} \right)^{-1} \ge \frac{1}{2y} &= \frac{1}{2(1 - \varepsilon)\sqrt{2\log\log(N^{n+1} - N^n)}}\\
        &= \frac{1}{2(1 - \varepsilon)\sqrt{2}}\times \frac{1}{\sqrt{\log (n\log N + \log (N-1))}}\\
        &\ge \frac{1}{2(1 - \varepsilon)\sqrt{2}}\times \frac{1}{\sqrt{\log(2n\log N) }}\\
        &= \frac{1}{2(1 - \varepsilon)\sqrt{2}}\times \frac{1}{\sqrt{\log2 + \log n + \log\log N}}\\
        &= \frac{1}{2(1 - \varepsilon)\sqrt{2}}\times \frac{1}{\sqrt{\log n + \log (2\log N)}}\\
      \end{align*}
      ここで、下から三行目の$\ge$について、全ての$n \ge 1$について、
      \[ n\log N + \log (N-1) \leq 2n\log N\]
      が成り立つことを用いている。実際、
      \begin{align*}
        2n\log N - n\log N - \log (N-1) &= n\log N - \log (N-1)\\
        &\ge 0
      \end{align*}
      である。

      加えて、$\beta \colon = [2\log N]$とすると、$n \ge \beta$のとき、
      \[\log n + \log 2\log N \leq 2\log n\]
      が成立する。実際、
      \begin{align*}
        2\log n - \log n - \log (2\log N) &= \log n - \log (2\log N)\\
        &\ge \log (2\log N) - \log (2\log N) \\
        &= 0
      \end{align*}

      したがって、$n \ge \max\{\alpha, \beta\} =\colon \gamma$のとき、続けて以下のように評価できる。
      \begin{align*}
        \left( y + \frac{1}{y} \right)^{-1} &\ge \frac{1}{2(1 - \varepsilon)\sqrt{2}}\times \frac{1}{\sqrt{\log n + \log 2\log N}}\\
        &\ge \frac{1}{2(1 - \varepsilon)\sqrt{2}}\times \frac{1}{\sqrt{2\log n }}\\
        &= \frac{1}{4(1 - \varepsilon)} \times \frac{1}{\sqrt{\log n }}\\
        &\ge \frac{1}{4(1 - \varepsilon)} \times \frac{1}{\log n}
      \end{align*}
      とできる。

      以上により、$n \ge \gamma$のとき、
      \[
        \frac{1}{\sqrt{2\pi}}\left( y + \frac{1}{y} \right)^{-1}\exp\left( -\frac{y^2}{2}\right) \ge \frac{1}{\sqrt{2\pi}} \frac{1}{4(1 - \varepsilon)}  \frac{1}{\log n} \frac{1}{(2\log N)^{(1-\varepsilon)^2}}  \frac{1}{n}
      \]
      $n$と関係がない部分をまとめて$C$と書くことにすると結局、
      \[
        \frac{1}{\sqrt{2\pi}}\left( y + \frac{1}{y} \right)^{-1}\exp\left( -\frac{y^2}{2}\right) \ge C \times \frac{1}{n\log n}
      \]
      と評価できる。

      よって、
      \begin{align*}
        &\sum_{n=1}^{\infty} \PP(F_n)\\
        \ge & \sum_{n=1}^{\infty} \frac{1}{\sqrt{2\pi}}\left( y + \frac{1}{y} \right)^{-1}\exp\left( -\frac{y^2}{2}\right)\\
        \ge & \sum_{n=1}^{\gamma - 1} \frac{1}{\sqrt{2\pi}}\left( y + \frac{1}{y} \right)^{-1}\exp\left( -\frac{y^2}{2}\right) + C \sum_{n=\gamma}^{\infty} \frac{1}{n\log n}\\
        = & \infty
      \end{align*}

  \subsection{141ページ上から4行目}
    Step2. で得られた式は
    \[
      \frac{S_k}{h(k)} \leq K
    \]
    であること。$(S_k)$はマルチンゲールであればよい。
    そこで$(-S_k)$というmartingale\footnote{$S_k$はmartingaleだから$-S_k$もmartingaleになる。}に$K=2$としてStep2.の結果を適用する。

  \subsection{141ページ上から8行目}
    \[
      \sup_{m \ge n}a_m \ge \sup_{m \ge n}a_{N^{m+1}}
    \]
    だから、
    \[
      \inf_{n}\sup_{m \ge n}a_m \ge \inf_{n}\sup_{m \ge n}a_{N^{m+1}}
    \]
    がなりたつ。
    \[a_n = \frac{S_k}{h(k)}\]
    と思えば、
    \[
      \limsup_n \frac{S_n}{h(n)} \ge \limsup_n \frac{S_{N^{n+1}}}{h(N^{n+1})}
    \]

  \subsection{141ページ上から9行目}
    \[
      \limsup_n \frac{S_{N^{n+1}}}{h(N^{n+1})} \ge \limsup_n \frac{(1 - \varepsilon)h(N^{n+1} - N^n) - 2h(N^n)}{h(N^{n+1})}
    \]
    である。
    \begin{align*}
      \limsup_n \frac{h(N^{n+1} - N^n)}{h(N^{n+1})} &= \limsup_n \sqrt{\frac{2(N^{n+1} - N^n)\log\log (N^{n+1} - N^n)}{2N^{n+1} \log\log N^{n+1}}}\\
      &= \limsup_n \sqrt{\left(1 - \frac{1}{N} \right)} \sqrt{\frac{\log\log (N^{n+1} - N^n)}{\log\log N^{n+1}}}\\
      &= \sqrt{\left(1 - \frac{1}{N} \right)} \limsup_n \sqrt{\frac{\log\log N^n(N-1)}{\log (n+1)\log N}}\\
      &= \sqrt{\left(1 - \frac{1}{N} \right)} \limsup_n \sqrt{\frac{\log (n\log N + \log (N-1))}{\log (n+1) + \log\log N}}\\
      &\ge \sqrt{\left(1 - \frac{1}{N} \right)} \limsup_n \sqrt{\frac{\log (n\log N )}{\log (n+1) + \log\log N}}\\
      &= \sqrt{\left(1 - \frac{1}{N} \right)} \limsup_n \sqrt{\frac{\log n + \log\log N }{\log (n+1) + \log\log N}}\\
      &\ge \sqrt{\left(1 - \frac{1}{N} \right)} \lim_n \sqrt{\frac{\log n + \log\log N }{\log (n+1) + \log\log N}}\\
      &= \sqrt{\left(1 - \frac{1}{N} \right)}
    \end{align*}
    最後の極限はロピタルの定理を用いた。一方、同様にすると、
    \begin{align*}
      \limsup_n \frac{h(N^n)}{h(N^{n+1})} &= \limsup_n \sqrt{\frac{2N^n\log\log N^n}{2N^{n+1} \log\log N^{n+1}}}\\
      &= \sqrt{\frac{1}{N}}\limsup_n \sqrt{\frac{\log\log N^n}{\log\log N^{n+1}}}\\
      &\ge \sqrt{\frac{1}{N}}
    \end{align*}

  \subsection{141ページ:14.8の命題の主張にある記号}
    この命題において$\varphi$と$\Phi$が用いられているが、
    $\varphi$は確率密度関数
    \[
      \varphi(x) = \frac{1}{\sqrt{2\pi}}e^{-\frac{1}{2}x^2}
    \]
    で、$\Phi$は累積分布関数である、すなわち、
    \[
      \Phi(x) = \int_{-\infty}^{x} \varphi(y) dy
    \]

  \subsection{141ページ下から2行目}
    \[
      \int_x^{\infty}y\varphi(y) dy = \int_x^{\infty}\Bigl(-\varphi^{\prime}(y)\Bigl) dy = - \int_x^{\infty}\varphi^{\prime}(y) dy =\varphi(x)
    \]
    であり、一方、
    \[
      \int_x^{\infty}y\varphi(y) dy \ge x\int_x^{\infty}\varphi(y) dy
    \]
    である。(b)も同様にできる。

  \subsection{142ページ下から2行目}
    仮定より
    \[
      c\PP(X\ge c) \leq \EE[Y;X\ge c]
    \]
    なので、この両辺に$pc^{p-2}$をかける。
    その後、両辺を$c$に関して積分すると教科書のような式が得られる。

  \subsection{143ページ上から7行目}
    "Suppose that $||Y||_p < \infty$"とあるが、$||Y||_p = \infty$の場合は明らかに主張の式は成立する。
    $||X||_q$の値についても考えなければならないが、ここでは$X$についても
    $||X||_q < \infty$と仮定して話を進めている。
    一般の$X$については証明の最後の三行で述べられる。

  \subsection{144ページ(a)の証明最後二行}
    条件付き期待値のイェンゼンの不等式より、
    \begin{align*}
      \EE[Z_r^p \mid \mathcal{F}_{r-1}] &\ge (\EE[Z_r \mid \mathcal{F}_{r-1}])^p\\
      &\ge (Z_{r-1})^p\\
      \EE[Z_r^p] &\ge \EE[Z_{r-1}^p]\\
      ||Z_r||_p &\ge ||Z_{r-1}||_p
    \end{align*}
    が得られるので、$||Z_r||_p$は$r$に関してa.s.で単調増加。
    $\mathcal{L}^p$収束しているので
    \[
      ||Z_{\infty}||_p = \lim_{r \to \infty}||Z_r||_p
    \]
    単調増加であることより、
    \[
      \lim_{r \to \infty}||Z_r||_p = \sup_r ||Z_r||_p
    \]

    (b)の証明は(a)を適用すればいい。

  \subsection{145ページ上1行目}
    "That $a_n \leq 1 $ follows from Jensen's inequality."
    とあるが、使い方に注意が必要。
    \[
      1 = \EE[X_n] = \EE \left[(\sqrt{X_n})^2 \right] \ge \EE \left[\sqrt{X_n} \right]^2 = a_n^2
    \]

  \subsection{145ページ上から7行目から11行目まで}
    \[
      \EE[N_n^2] \leq \frac{1}{\Pi a_n}^2 < \infty
    \]
    が得られたが、これにより、$\displaystyle{\left(\frac{1}{\Pi a_n}\right)^2}$
    が上界であることが分かるので、
    \[
      \sup \EE[N_n^2] < \infty
    \]
    が導かれるという話。

    Doob's $\mathcal{L}^p$不等式を$p=2$で適用し両辺を二乗すると、
    \[
      \EE[\sup N_n^2] \leq 4\sup\EE[N_n^2] < \infty
    \]
    が得られ、一方で$N_n^2,M_n$の定義から各$n$について
    \[
      M_n \leq N_n^2
    \]
    なので、
    \[
      \EE[\sup M_n] \leq \EE[\sup N_n^2]
    \]
    したがって合わせて
    \[
      \EE[\sup M_n] \leq 4\sup\EE[N_n^2] < \infty
    \]
    がわかる。

    あとは13.3(b),13.7より、$(iii),(ii),(i)$の順で命題が証明できる。

  \subsection{14.12 $(v) \Rightarrow (iv)$の証明について}
    40ページ一番下のExcerciseをみて。
  \subsection{14.12 $(i) \Rightarrow (iv)$の証明について}
    背理法。$(i)$かつ$(iv)$ではないと仮定する。
    $(i)$が成り立つことより、
    \[
      \EE[M_{\infty}] = 1 = \lim\EE[M_n]
    \]
    すなわち、
    \[
      \EE[M_{\infty}] = \lim\EE[M_n]
    \]
    が成立する。すると、
    \[
      \Pi a_n = 0
    \]
    なので、独立性と上で示したことより、
    \[
      \EE \left[ \sqrt{M_{\infty}} \right] = 0
    \]
    と書き換えることができる。
    よって、$M_{\infty} = 0$でなければならない。しかしこれは$(i)$に反する。

  \subsection{146ページ証明中の(b)について}
    この命題は論理式で書くと
    \[
      \forall \varepsilon >0, \exists \delta >0, \,\,s.t.\,\,\forall F \in \mathcal{F}, \PP(F) < \delta \Rightarrow \mathbb{Q}(F)<\varepsilon
    \]
    これは背理法で示す。否定命題
    \[
      \exists \varepsilon >0, \forall \delta >0, \,\,s.t.\,\,\exists F \in \mathcal{F}, \PP(F) < \delta \text{かつ} \mathbb{Q}(F)\ge\varepsilon
    \]
    を仮定する。
    この否定命題によって存在が保証されている$\varepsilon_0$を一つ選んで固定する。
    $\varepsilon_0$と$\frac{1}{2}$に対してある集合$F_1$が取れる。
    再び、$\varepsilon_0$と$\frac{1}{2^2}$に対してある集合$F_2$が取れる。
    繰り返すが、$\varepsilon_0$と$\frac{1}{2^3}$に対してある集合$F_3$が取れる。
    このようにして、ある集合列$(F_n)$があって、
    \[
      \PP(F_n) < \frac{1}{2^n} \text{かつ} \mathbb{Q}(F_n) \ge \varepsilon_0
    \]
    となる。
    \[
      \sum\PP(F_n) \leq \sum\frac{1}{2^n} < \infty
    \]
    より、(BC1)が使えて、
    \[
      \PP(\limsup F_n) =0
    \]
    であることが分かる。一方絶対連続性より、$\mathbb{Q}(\limsup F_n) = 0$
    である。
    しかし、$\mathbb{Q}$は有限測度だから、
    \[
      \mathbb{Q}(\limsup F_n) \ge \limsup\mathbb{Q}(F_n) \ge \varepsilon_0
    \]
    これは矛盾。

  \subsection{146ページ証明の上から5行目から13行目}
    $N =\{1,2,\cdots ,n\}$とする。$2^N =\{I_m ; 1 \leq m \leq 2^n\}$とする。
    $1 \leq m \leq 2^n$について、
    \[A_{n,I_m} := \left( \bigcap_{i \in I_m}F_i \right) \cap \left( \bigcap_{i \in N - I_m}F_i^c \right)\]
    と定義する。このとき、すべての$m$について、
    \[
      A_{n,I_m} \in \mathcal{F}_n
    \]
    であり、同じくすべての$i \neq j$なる$i,j$について、
    \[
      A_{n,I_i} \cap A_{n,I_j} = \emptyset
    \]
    である。また、
    \[
      \bigcup_{m=1}^{2^n}A_{n,I_m} = \Omega
    \]
    が成立する。すなわち、集合族$\{A_{n,I_m} \mid 1 \leq m \leq 2^n\}$は
    $\Omega$の有限分割になっている。
    実際、$\omega \in \Omega$を取ってみると、すべての$i \in N$について、
    \[
      \omega \in F_i \quad\text{または}\quad \omega \in F_i^c
    \]
    である。そこで、
    \begin{align*}
      N_1 &= \{i \in N \mid \omega \in F_i\}\\
      N_2 &= \{i \in N \mid \omega \in F_i^c\} = N - N_1
    \end{align*}
    とおくと、$N_1 \in 2^N$なので、ある$m^{\prime}$があって、
    \[
      N_1 = I_{m^{\prime}}
    \]
    とかける。
    さて、
    \[
      A_{n,I_{m^{\prime}}} = \left( \bigcap_{i \in I_{m^{\prime}}}F_i \right) \cap \left( \bigcap_{i \in N - I_{m^{\prime}}}F_i^c \right) \ni \omega
    \]
    なので、
    \[
      \omega \in \bigcup_{m=1}^{2^n}A_{n,I_m}
    \]
    ゆえに
    \[
      \bigcup_{m=1}^{2^n}A_{n,I_m} \supset \Omega
    \]
    である。逆の包含関係は自明。

    さて、ここで、$A_{n,I_1},A_{n,I_2},\cdots,A_{n,I_{2^n}}$のうちいくつかは空集合になって同じになることがある。
    そこで空集合は省いて、空でない集合に改めて添え字をつけて
    \[
      A_{n,1},A_{n,2},\cdots,A_{n,r_n}
    \]
    とする。$r_n \leq 2^n$である。これらの集合はどの二つをとっても互いに交わらない。
    \[
      \mathcal{G} := \left\{ G \,\, \middle| \exists I \subset \{1,2,\cdots, r_n\}, \,\, G =\bigcup_{i \in I}A_{n,i} \right\}
    \]
    とする。また、表記の簡単さのために、$N_r := \{1,2,\cdots,r_n\}$とおく。
    \begin{claim*}
      $\mathcal{G} = \mathcal{F}_n$
    \end{claim*}

    \begin{enumerate}
      \item[Step1.] $\mathcal{G}$がalgebraであること。
        \begin{itemize}
          \item $\displaystyle{\bigcup_{i=1}^{r_n} A_{n,i} = \Omega}$より、$\Omega \in \mathcal{G}$.
          \item $G \in \mathcal{G}$とすると、$\mathcal{G}$の定義よりある$I \subset N_r$があって、
            \[
              G = \bigcup_{i \in I}A_{n,i}
            \]
            このとき、
            \[
              G^c = \Omega - \bigcup_{i \in I}A_{n,i} = \bigcup_{i \in (N_r - I)}A_{n,i}
            \]
            $(N_r - I) \subset N$なので、$G^c \in \mathcal{G}$である。
          \item $G_1,\cdots,G_m \in \mathcal{G}$とする。
            このとき、$\mathcal{G}$の定義より、
            $I_1,\cdots,I_m \subset N_r$があって、
            \[
              G_1 = \bigcup_{i \in I_1}A_{n,i}, \cdots, G_m = \bigcup_{i \in I_m}A_{n,i}
            \]
            と書ける。そこで、$I = I_1 \cup \cdots \cup I_m$とおくと、
            \[
              \bigcup_{i=1}^m G_i = \bigcup_{i \in I}A_{n,i}
            \]
            と書けるので、$G \in \mathcal{G}$
        \end{itemize}
        以上により、$\mathcal{G}$がalgebraであることが示された。
      \item[Step2.] $\mathcal{G} \subset \mathcal{F}_n$であること。
        実際、各$A_{n,i}$は$A_{n,i} \in \mathcal{F}_n$である。
        よって、$\forall G \in \mathcal{G}$は$G \in \mathcal{F}_n$である。
      \item[Step3.] $\mathcal{F}_n \subset \mathcal{G}$であること、
      すなわち、$F_1,\cdots,F_n \in \mathcal{G}$であること。
      $F_1 \in \mathcal{G}$であることだけを示せばよい。
      \[
        F_1 = \bigcup_{I \subset N-\{1\}} \left( F_1 \cap \left( \left(\bigcap_{i \in I}F_i \right) \cap \left(\bigcap_{j \in N-I-\{1\}}F_j^c\right) \right) \right)
      \]
      とかける。
      (左辺)$ \supset $(右辺)は明らか。$\omega \in F_1$を任意に取る。
      当然、
      \[
        \omega \in \Omega = \bigcup A_{n,I_m}
      \]
      である。
      なので、ある$I_m \in 2_N$があって、
      \[
        \omega  \in A_{n,I_m}
      \]
      とできる。
      このとき、必ず$1 \in I_m$である。
      もし、そうでないとすると、$1 \in I_m^c$となるので、
      \[
        \omega \in F_1^c
      \]
      が導かれてしまうので、矛盾する。
      したがって、$I = I_m - \{ 1 \}$とすると、
      \[
        \omega \in  F_1 \cap \left( \left(\bigcap_{i \in I}F_i \right) \cap \left(\bigcap_{j \in N-I-\{1\}}F_j^c\right) \right)
      \]
      よって、$\omega \in \text{(右辺)}$。
      よって、(左辺)$ = $(右辺)

      したがって、$\mathcal{G}$の定義より$F_1 \in \mathcal{G}$
    \end{enumerate}
    以上により、$\mathcal{G} = \mathcal{F}_n$であることが示された。

  \subsection{146ページ下から2行目}
    "$X_n \in \mathcal{L}^1(\Omega, \mathcal{F}_n, \PP)$"とある。
    実際に計算して確かめる。ここで$r_n$はアトムの個数である。前小節参照。
    $X_n$の定義より、$X_n$は単関数なので可測関数。
    $\mathbb{P}\gg\mathbb{Q}$であること、すなわち$\PP(A_{n,k}) = 0 \Rightarrow \mathbb{Q}(A_{n,k}) = 0$であることに気を付ければ、
    \begin{align*}
      \int_{\Omega} X_n d\PP &= \sum_{k=1,\PP(A_{n,k})\neq 0}^{r_n} \frac{\mathbb{Q}(A_{n,k})}{\PP(A_{n,k})}\PP(A_{n,k}) + \sum_{k=1,\PP(A_{n,k})= 0}^{r_n} 0 \times \PP(A_{n,k})\\
      &= \sum_{k=1,\PP(A_{n,k})\neq 0}^{r_n} \mathbb{Q}(A_{n,k}) + 0\\
      &= \sum_{k=1,\PP(A_{n,k})\neq 0}^{r_n} \mathbb{Q}(A_{n,k}) + \sum_{k=1,\PP(A_{n,k})= 0}^{r_n} \mathbb{Q}(A_{n,k})\\
      &= \mathbb{Q}(\Omega) < \infty
    \end{align*}
    $\PP(A_{n,k})= 0$になるかどうかで計算方法が変わってくるので、注意しよう。

  \subsection{146ページ一番下の行}
    (c)の式が成り立つことの証明。

    $F \in \mathcal{F}_n$を任意にとる。すると、ある$I \subset N_r$\footnote{$N_r$は二つ前の小節を参照。$N_r = \{ 1, 2, \cdots,n_r \}$}
    が存在して、
    \[
      F = \bigcup_{i \in I}A_{n,i}
    \]
    と書ける。よって再び$\PP \gg \mathbb{Q}$であることに注意すると
    \begin{align*}
      \EE[X_n;F] &= \int_F X_n d\PP \\
      &= \int_{\bigcup_{i \in I}A_{n,i}} X_n d\PP\\
      &= \sum_{i\in I}\int_{A_{n,i}}X_n d\PP\\
      &= \sum_{i\in I, \PP(A_{n,i})\neq0}\mathbb{Q}(A_{n,i}) + 0\\
      &= \sum_{i\in I, \PP(A_{n,i})\neq0}\mathbb{Q}(A_{n,i}) + \sum_{i\in I, \PP(A_{n,i})=0}\mathbb{Q}(A_{n,i})\\
      &=\mathbb{Q}(F)
    \end{align*}
    となる。

  \subsection{147ページ上から2行目}
    "$X$ is a martingale relative to the filtration..."のところ。
    条件付き期待値の定義よりmartingaleであることの三つ目の条件を確かめることができる。
    $Y$を$\EE[X_{n+1} \mid \mathcal{F}_{n}]$の一変形とする。
    $F \in \mathcal{F}_n$をとる。$F \in \mathcal{F}_n$ならば$F \in \mathcal{F}_{n+1}$であることに気を付ければ次のように計算できる。
    \begin{align*}
      \int_F Y d\PP = \int_F X_{n+1} d\PP = \int_{\bigcup_{j \in J}A_{n,j}}X_{n+1}d\PP\\
      = \mathbb{Q}(F) = \int_F X_n d\PP
    \end{align*}
    よって、条件付き期待値の定義から
    \[
      \EE[X_{n+1} \mid \mathcal{F}_{n}] = Y = X_n
    \]

  \subsection{147ページ上から12行目}
    "$X_n \longrightarrow X \quad \text{in} \quad\mathcal{L}^1$"について。
    概収束はしているので確率収束している。
    これとUIであることから13.7の内容を使う。

  \subsection{147ページ上から14行目}
    14.1より、
    \[
      \EE[X \mid \mathcal{F}_n] = X_n
    \]
    がなりたつ。これに気を付けるだけ。

  \subsection{ラドンニコディムの定理一意性の証明}
    これは可分の条件が不要。

    $X,Y \in \mathcal{L}^1(\Omega, \mathcal{F}, \PP)$を
    主張を満たす確率変数とする。
    $F \in \mathcal{F}$とすると、
    \begin{align*}
      \mathbb{Q}(F) = \int_F X d\PP = \EE[X;F]\\
      \mathbb{Q}(F) = \int_F Y d\PP = \EE[Y;F]
    \end{align*}
    したがって、
    \[
      Y = \EE[X \mid \mathcal{F}]
    \]
    が分かる。ところでこの右辺の$X$は$\mathcal{F}$可測なので
    $\EE[X \mid \mathcal{F}] = X$
    よって、
    \[
      X = Y \quad (a.s.)
    \]

  \subsection{ラドンニコディムの定理:一般の場合}
    この部分は、証明をすべて書きます。教科書の証明では
    (e)$\Rightarrow$(d)をまず証明し、その次に(e)を証明した後、
    (d)$\Rightarrow$(II)、という流れで証明しています。
    順番が前後しているので、このpdfでは、
    (e),(e)$\Rightarrow$(d),(d)$\Rightarrow$(II)
    、の順番で証明します。
    \subsubsection{(e)が成り立つこと}
      背理法で示す。(e)でないとする。すなわち、
      \[
        \exists \varepsilon > 0 ,\forall \mathcal{K} \in \mathbf{Sep},\exists \mathcal{G}_1,\exists\mathcal{G}_2 \in \mathbf{Sep} \,\, s.t. \,\, \mathcal{K} \subset \mathcal{G}_1,\mathcal{G}_2 \text{\,\,and\,\,} ||X_{\mathcal{G}_1} - X_{\mathcal{G}_2}||_1 \ge \varepsilon
      \]
      が成立しているとする。
      この仮定で存在が保証されている$\varepsilon$を任意に一つ固定する。
      $\varepsilon_0 := \frac{1}{2}\varepsilon$とする。
      $\mathcal{K} \in \mathbf{Sep}$を任意に一つ選ぶ。
      仮定の命題より$\mathcal{G}_1, \mathcal{G}_2 \in \mathbf{Sep}$
      が存在して、$\mathcal{K} \subset \mathcal{G}_1,\mathcal{G}_2$であり、
      \begin{align*}
        ||X_{\mathcal{G}_1} - X_{\mathcal{G}_2}||_1 &\ge \varepsilon = 2\varepsilon_0\\
        \EE[|X_{\mathcal{G}_1} - X_{\mathcal{G}_2}|] &\ge 2\varepsilon_0\\
        \EE[|X_{\mathcal{G}_1} - X_{\mathcal{K}}|] + \EE[ X_{\mathcal{K}} - X_{\mathcal{G}_2}|] &\ge 2\varepsilon_0\\
        ||X_{\mathcal{G}_1} - X_{\mathcal{K}}||_1 + ||X_{\mathcal{G}_2} - X_{\mathcal{K}}||_1 &\ge 2\varepsilon_0
      \end{align*}
      したがって、
      \[
        \max \{||X_{\mathcal{G}_1} - X_{\mathcal{K}}||_1, ||X_{\mathcal{G}_2} - X_{\mathcal{K}}||_1\} \ge \varepsilon_0
      \]
      そこで、
      \[
        \mathcal{K}(1) := \text{arg max}_{\mathcal{G}_1,\mathcal{G}_2} \{||X_{\mathcal{G}_1} - X_{\mathcal{K}}||_1, ||X_{\mathcal{G}_2} - X_{\mathcal{K}}||_1\}
      \]
      とする。\footnote{argmaxのコマンドは編集の都合上用意しませんでした。数式環境の中でtextコマンドを使っています。}
      つまり、$\mathcal{K}(1)$は
      $||X_{\mathcal{G}_1} - X_{\mathcal{K}}||_1, ||X_{\mathcal{G}_2} - X_{\mathcal{K}}||_1$
      の二つのノルムのうち、値が大きくなる方の$\mathcal{G}$をえらんでいる。
      $\mathcal{K}(1)$に対してもう一度同じ操作を行うと$\mathcal{K}(2)$が取れる。
      当然$\mathcal{K}(2)$は、$\mathcal{K}(1) \subset \mathcal{K}(2)$であり、
      $||X_{\mathcal{K}(2)} - X_{\mathcal{K}(1)}||_1 \ge \varepsilon_0$を満たしている。
      この手続きを繰り返せば、可分な$\sigma$-alg.
      の増大列$\mathcal{K}(1) \subset \mathcal{K}(2) \subset \cdots$で
      全ての$n \in \NN$に対して
      \[
        ||X_{\mathcal{K}(n+1)} - X_{\mathcal{K}(n)}||_1 > \frac{1}{2}\varepsilon_0
      \]
      を満たしているものの存在が分かる。

      さて、こうして得られた$\sigma$-alg.の列$(\mathcal{K}(n))_{n \in \NN}$
      をフィルトレーションとする。
      このとき、$(X_{\mathcal{K}(n)})_{n \in \NN}$は
      このフィルトレーションに関してUI martingaleになる。
      martingaleになることは易しい。adopted processであることと可積分であることは、
      ラドンニコディムの定理より$X_{\mathcal{K}(n)} \in \mathcal{L}^1(\Omega, \mathcal{K}(n), \PP)$
      であることから分かる。
      martingaleであることの三つ目の条件はラドンニコディム微分である$X_{\mathcal{K}(n)}$の性質を用いる。
      つまり、$F \in \mathcal{K}(n)$を任意にとると、
      $F \in \mathcal{K}(n+1)$であることから、
      $X_{\mathcal{K}(n+1)}$を$F$上で積分することができるようになる。
      そこで実際に計算を行おうとすると次のようなことが分かる。
      \begin{align*}
        \EE[X_{\mathcal{K}(n+1)} ; F] = \int_F X_{\mathcal{K}(n+1)}d\PP = \mathbb{Q}(F)\\
        \EE[X_{\mathcal{K}(n)} ; F] = \int_F X_{\mathcal{K}(n)}d\PP = \mathbb{Q}(F)
      \end{align*}
      したがって、条件付き期待値の定義より、
      \[
        \EE[X_{\mathcal{K}(n+1)} \mid \mathcal{K}(n)] = X_{\mathcal{K}(n)}
      \]
      であることが分かる。
      UIであることは13.1(a)の命題を使う。
      $\varepsilon > 0$を任意に一つ選ぶ。
      13.1(a)より$\delta > 0$が存在して、
      \[
        \PP(F) < \delta \Rightarrow E[X;F] < \varepsilon
      \]
      となる。そこで、$K\mathbb{Q}(\Omega) < \delta$となる$K$をとると、
      \[
        \PP(X_{\mathcal{K}(n)} > K) \leq \frac{1}{K} \EE[X_{\mathcal{K}(n)}] = \frac{1}{K}\mathbb{Q}(\Omega) < \delta
      \]

      以上により、$(X_{\mathcal{K}(n)})_{n \in \NN}$はある$X$に$\mathcal{L}^1$収束することが分かる。
      しかし、これが「$(e)$でない」という仮定に矛盾する。
      実際、$(e)$でないことより、ある正の数$\varepsilon_0$が存在して
      \[
        ||X_{\mathcal{K}(n)} - X_{\mathcal{K}(n+1)}||_1 > \varepsilon_0
      \]
      となる$\mathbf{Sep}$の増大列$(\mathcal{K}(n))_{n \in \NN}$がとれる。
      しかし、$(X_{\mathcal{K}(n)})_{n \in \NN}$が
      $\mathcal{L}^1$収束することから、この$\varepsilon_0$に対して
      ある自然数$N_{\varepsilon_0}$が存在して
      すべての$n (\ge N_{\varepsilon_0})$について
      \[
        ||X - X_{\mathcal{K}(n)}||_1 < \frac{\varepsilon_0}{2}
      \]
      となる。
      よって次の計算ができる。
      \begin{align*}
        ||X_{\mathcal{K}(N_{\varepsilon_0} + 1)} - X_{\mathcal{K}(N_{\varepsilon_0})}||_1 > \varepsilon_0\\
        ||X_{\mathcal{K}(N_{\varepsilon_0} + 1)} - X||_1 +  ||X - X_{\mathcal{K}(N_{\varepsilon_0})}||_1 > \varepsilon_0\\
        2 \times \frac{\varepsilon_0}{2} > \varepsilon_0
      \end{align*}
      よって矛盾が生じる。

    \subsubsection{(e)$\Rightarrow$(d)がなりたつこと}
      (e)が成り立つ。
      $\mathcal{K}_n \in \mathbf{Sep}$を
      \[
        \mathcal{K}_n \subset \forall \mathcal{G}_1, \forall \mathcal{G}_2 \in \mathbf{Sep} \Rightarrow ||X_{\mathcal{G}_1} - X_{\mathcal{G}_2}||_1 < 2^{-(n+1)}
      \]
      を満たすものとする。これは(e)成立しているから実際にそのような$\mathcal
      {K}_n$を選ぶことができる。

      さて、$\mathcal{H}(n) = \sigma(\mathcal{K}(1), \cdots, \mathcal{K}(n))$とする。
      $\mathcal{H}(n) \in \mathbf{Sep}$である。
      このとき、$X := \lim_{n \to \infty}X_{\mathcal{H}(n)}$が
      a.s.の意味でも$\mathcal{L}^1$の意味でも存在する。
      このことの証明は教科書の6.10(a)に倣(なら)う。

      数列$(k_n)_{n \in \NN}$を単調増加で$k_n \longrightarrow \infty$であり、すべての$n$について$k_n \ge n$であるようなものとする。
      $r,s \ge k_n$とすると、$r,s \ge n$となるため、
      $\mathcal{H}(r),\mathcal{H}(s) \supset \mathcal{H}(n)$となる。
      したがって、
      \[
        ||X_{\mathcal{H}(r)} - X_{\mathcal{H}(s)}||_1 < 2^{-(n+1)}
      \]
      が成立する。特に$r = k_{n+1}, s = k_n$と思えば
      \[
        ||X_{\mathcal{H}(k_{n+1})} - X_{\mathcal{H}(k_n)}||_1 < 2^{-(n+1)}
      \]
      左辺を定義によって書き換えると、
      \[
        \EE[|X_{\mathcal{H}(k_{n+1})} - X_{\mathcal{H}(k_n)}|] < 2^{-(n+1)}
      \]
      (MON)より、
      \[
        \EE\left[\sum_{n=1}^{\infty}|X_{\mathcal{H}(k_{n+1})} - X_{\mathcal{H}(k_n)}|\right] < \sum_{n=1}^{\infty}2^{-(n+1)} < \infty
      \]
      よって、$\displaystyle{\sum_{n=1}^{\infty}|X_{\mathcal{H}(k_{n+1})} - X_{\mathcal{H}(k_n)}|}$
      はa.s.で有限である。
      ここで、
      \[
        \mathcal{N} = \left\{\omega \;\middle|\; \sum_{n=1}^{\infty}|X_{\mathcal{H}(k_{n+1})}(\omega) - X_{\mathcal{H}(k_n)}(\omega)| = \infty \right\}
      \]
      とする。当然$\PP(\mathcal{N}) = 0$である。
      確率変数$X$を次のように定義する。
      \[
        X(\omega)= \begin{cases}
          X_{\mathcal{H}(k_1)}(\omega) + \sum_{n=1}^{\infty} \left(X_{\mathcal{H}(k_{n+1})}(\omega) - X_{\mathcal{H}(k_n)}(\omega) \right) & [\text{if}\quad\omega \in \mathcal{N}^c]\\
          0 & [\text{if}\quad\omega \in \mathcal{N}]
      \end{cases}
      \]
      すると明らかに
      \[
        \lim_{n \to \infty}X_{\mathcal{H}(n)} = X \quad \text{(a.s.)}
      \]
      である。

      次に$\mathcal{L}^1$の意味でも$X$に収束することを示す。
      $\varepsilon >0$を任意に一つ固定する。
      自然数$N_{\varepsilon}$を$\displaystyle{\frac{1}{2^{N_{\varepsilon}+1}} < \varepsilon \leq \frac{1}{2^{N_{\varepsilon}}}}$を満たすものとする。
      このとき、$n,m \ge N_{\varepsilon}$なら、
      $\mathcal{H}(n),\mathcal{H}(m) \supset \mathcal{H}(N_{\varepsilon})$
      なので、
      \[
        ||X_{\mathcal{H}(n)} - X_{\mathcal{H}(m)}||_1 < 2^{-(N_{\varepsilon}+1)} < \varepsilon
      \]
      となる。$r (\ge N_{\varepsilon})$を一つ固定すると、
      \begin{align*}
        \EE[|X_{\mathcal{H}(r)} - X|] &= \EE[\lim_{n \to \infty}|X_{\mathcal{H}(r)} - X_{\mathcal{H}(k_n)}|]\\
        &= \EE[\liminf_{n \to \infty}|X_{\mathcal{H}(r)} - X_{\mathcal{H}(k_n)}|]\\
        &\leq \liminf_{n \to \infty}\EE[|X_{\mathcal{H}(r)} - X_{\mathcal{H}(k_n)}|]\\
        &= \liminf_{n \to \infty}||X_{\mathcal{H}(r)} - X_{\mathcal{H}(k_n)}||_1 \\
        &< \varepsilon
      \end{align*}
      よって、$\mathcal{L}^1$の意味でも収束する。これで(d)を証明することができある。
      全ての$n(\ge N_{\varepsilon})$と$(\mathcal{H}(n) \subset \;)\mathcal{G} \in \mathbf{Sep}$について、
      \begin{align*}
        ||X_{\mathcal{G}} - X||_1 &\leq ||X_{\mathcal{G}} - X_{\mathcal{H}(n)}||_1 + ||X_{\mathcal{H}(n)} - X||_1\\
        &< 2^{-(N_{\varepsilon} + 1)} + 2^{-(N_{\varepsilon} + 1)}\\
        &< 2\varepsilon
      \end{align*}
    \subsubsection{(d)$\Rightarrow$(II)がなりたつこと}
      (d)が成立する。$\varepsilon > 0 $を固定。
      (d)より、ある$\mathcal{K} \in \mathbf{Sep}$が存在して、
      全ての$(\mathcal{K} \subset \;)\mathcal{G} \in \mathbf{Sep}$
      について
      \[||X_{\mathcal{G}} - X||_1 < \varepsilon\]
      となる。
      いま、$F \in \mathcal{F}$を任意にとる。
      この$F$について当然$\sigma(\mathcal{K},F) \in \mathbf{Sep}$
      であり、$\mathcal{K} \subset \sigma(\mathcal{K},F)$だから
      $||X - X_{\sigma(\mathcal{K},F)}||_1 < \varepsilon $である
      一方、
      \begin{align*}
        |\EE[X;F] - \mathbb{Q}(F)| &= |\EE[X - \mathbb{Q}(F) ; F]\\
        &\leq \EE[|X - \mathbb{Q}(F)| ; F]\\
        &\leq ||X - X_{\sigma(\mathcal{K},F)}||_1\\
        &< \varepsilon
      \end{align*}
      $\varepsilon \rightarrow 0$とすれば定理が証明される。

  \subsection{149ページ下から7行目}
    "We say that $Y$ is (a version of)..." の$Y$はおそらく$X$の誤植。そうじゃないと話が通じない。
  \subsection{149ページ下から7,8行目}
    "Then $\PP$ is absolutely continuous to $\mathbb{Q}$ if and only if $\PP(X>0) = 1$,
    and then $X^{-1}$ is a version of $d\PP/d\mathbb{Q}$ "について。

    \begin{proof}
      ($\Rightarrow$) $\QQ \gg \PP$ とする。
      仮定より、$\PP \gg \QQ$なので、
      \[
        \PP(F) = 0 \Rightarrow \QQ(F) = 0
      \]
      $\QQ \gg \PP$なので、
      \[
        \QQ(F) = 0 \Rightarrow \PP(F) = 0
      \]
      したがって、
      $\PP(F) = 0 \Leftrightarrow \QQ(F) = 0$である。

      $\QQ(F) = \EE[X; F]$なので、
      \[
        \QQ(\{X \leq 0\}) = \EE[X; X \leq 0]
      \]
      $\QQ$は測度より、
      \[
        0 \leq \QQ(\{X \leq 0\}) = \EE[X; X \leq 0]
      \]
      一方、
      \[
        X1_{\{X \leq 0\}} \leq 0
      \]
      なので、
      \[
        \QQ(\{X \leq 0\}) = \EE[X; X \leq 0] \leq 0
      \]
      よって、
      \[
        \QQ(X \leq 0) = 0
      \]
      よって、$\PP(X \leq 0) = 0$となるから、$\PP(X > 0) = 1$である。

      ($\Leftarrow$) $\PP(X>0) = 1$とする。
      $F \in \mathcal{F}$で$\QQ(F) = 0$とする。
      \[
        \PP(F) = \PP(F \cap \{X \leq 0\}) + \PP(F \cap \{X>0\})
      \]
      だが、仮定より$\PP(X \leq 0) = 0$なので、
      $F \cap \{X \leq 0 \} \subset \{X \leq 0\}$より、
      \[
        \PP(F \cap \{X \leq 0\}) = 0
      \]
      したがって、
      \[
        \PP(F) = \PP(F \cap \{X > 0\})
      \]
      一方、$\QQ(F) = 0$より、$\QQ(F \cap \{X > 0\}) = 0$である。
      よって、
      \[
        \int X1_{F \cap \{X > 0\}} d\PP = 0
      \]
      $X1_{F \cap \{X > 0\}} > 0$であり、その積分の値が$0$であるから、
      \[
        \PP(F \cap \{X > 0\}) = 0
      \]
      よって、$\PP(F) = 0$。したがって、$\QQ \gg \PP$

      さて、$\PP(X > 0) = 1$のとき、
      \[
      Y(\omega) = \begin{cases}
        \frac{1}{X(\omega)} & (X(\omega) > 0)\\
        0 & (X(\omega) \leq 0)
      \end{cases}\]
      とすると、$Y \in (m\mathcal{F})^+$である。
      ラドンニコディムの定理より、$\QQ = X\PP$とできて、
      5.14(d)より
      \[
        Y\QQ = Y(X\PP)
      \]
      仮定より、これはalmost surelyで以下のように変形できる。
      \begin{align*}
        Y\QQ &= \frac{1}{X}\QQ \quad (a.s.)\\
        Y(X\PP) &= \frac{1}{X}(X\PP) \quad (a.s.)\\
        &= \left(\frac{1}{X}X\right)\PP \quad (a.s.)\\
        &=\PP \quad (a.s.)
      \end{align*}
      よって、almost surely で
      \[
        \frac{1}{X}\QQ = \PP
      \]
      である。
    \end{proof}

  \subsection{150ページ下から13行目}
    全ての$F \in \mathcal{F}_n$について
    $\QQ(F) = \EE[Y_1Y_2\cdots Y_n;F]$を言えばよい。
    しかし、$\mathcal{F}$上で一致することを言うのは困難なので、
    $\pi$-system上で考える。
    つまり、$\bigcup_{n=1}^{\infty}\mathcal{F}_n$上で
    一致することを言えばよい。それは即ち、
    任意の$n$について$\mathcal{F}_n$上で一致することを言えばよいことになる。
    これを言うためには$\mathcal{F}_n$を生成する$\pi$-systemである、
    $\{(\infty,a] \mid a \in \RR\}$上で一致することを言えばよい。
    $F\in \{(\infty,a_1] \times \cdots \times (\infty,a_n] \mid a \in \RR\}$を任意にとると、
    \begin{align*}
      \text{(右辺)} &= \EE[Y_1Y_2\cdots Y_n ;F]\\
      &= \EE[I_F r_1(X_1)r_2(X_2)\cdots r_n(X_n)]\\
      &= \int_{\RR}\cdots \int_{\RR} I_F(\omega_1,\omega_2,\cdots ,\omega_n)r_1(X_1(\omega_1,\omega_2,\cdots ,\omega_n))\cdots r_n(X_n(\omega_1,\omega_2,\cdots ,\omega_n))d\omega_n \cdots d\omega_1 \\
      &= \int_{\RR}\cdots \int_{\RR} I_F(\omega_1,\omega_2,\cdots ,\omega_n)g_1(\omega_1)\cdots g_n(\omega_n)d\omega_n \cdots d\omega_1 \\
      &= \QQ(F)
    \end{align*}

  \subsection{151ページ上から11行目}
    "in which case $\PP = \QQ$."について。

    $f = g$(a.e)の時密度関数がalmost surelyで等しいので、
    全ての$n \in \NN$について
    $F \in \{(\infty,a_1] \times \cdots \times (\infty,a_n] \mid a \in \RR\}$をとると、
    \begin{align*}
      \PP(F) &= \int_{\RR}\cdots \int_{\RR}I_F(\omega_1,\cdots ,\omega_n)f(\omega_1,\cdots \omega_n)d\omega_1 \cdots d\omega_n\\
      &= \int_{\RR} \cdots \int_{\RR}I_F(\omega_1,\cdots ,\omega_n)g(\omega_1,\cdots \omega_n)d\omega_1 \cdots d\omega_n\\
      &= \QQ(F)
    \end{align*}
    よって、$\pi$-system上で一致したので、結局$\PP = \QQ$となる。

  \subsection{151ページ下から7行目}
    (b)の条件について。
    (b)の$[M]_{\infty} = \uparrow \lim M_n$はおそらく、
    $[M]_{\infty} = \uparrow \lim [M]_n$の誤植。
    そうでないと後に登場する$C \bullet M \in \mathcal{H}_0^1$が証明できない。\footnote{これ日本語訳でも治ってなかったけど、もしかしたら誤植ではない?これが正しい表記?}

  \subsection{152ページ(e)の二行下の式}
    一つ目の$=$は(9.7.a)より。
    $\sigma(M)$の条件付き期待値ににすることで、$M$が分かっているとき、とみなせる。
    よって、以降の計算では$M$を既知として計算を進めている。

    $v_n,W_n$は定義に従って計算すると、
    それぞれ$[M]_n,(\varepsilon\bullet M)_n$と対応していることが分かる。
