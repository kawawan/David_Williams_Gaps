%% 担当 @7pitps

\section{Chapter 1}
    \subsection{p16 Exercise}
        $C_1\cap C_2\not\in\mathcal{C}$となる$C_1, C_2\in\mathcal{C}$の具体例を示す。
        $a_n(C) := \frac{\#\{k\in C; k\leq n\}}{n}$
        とする。このとき、$2\NN\in\mathcal{C}$であることを示す。
        \begin{align*}
            a_n(2\NN) = \left \{
            \begin{array}{cc}
                \frac{1}{2} & n\in 2\NN\\
                \frac{(n-1)}{2 n} & \mathrm{else}
            \end{array}
            \right.
        \end{align*}
        となるので、
        \begin{align*}
            \sup_{n\leq k} a_k(2\NN) &= \frac{1}{2} \\
            \inf_{n\leq k} a_k(2\NN) &= \frac{n-1}{2n}
        \end{align*}
        で、それぞれの極限は
        \begin{align*}
            \lim_{n\to\infty}\sup_{n\leq k} a_k(2\NN) &= \frac{1}{2} \\
            \lim_{n\to\infty}\inf_{n\leq k} a_k(2\NN) &= \lim_{n\to\infty}\frac{n-1}{2n} = \frac{1}{2}
        \end{align*}
        すなわち、
        $\lim_{n\to\infty} a_n(2\NN) = \frac{1}{2}$
        となり収束するので、
        $2\NN\in\mathcal{C}$。

        次に、$C$を次のように構成する。$n = 0,1,2,\ldots$について、
        \begin{itemize}
            \item $[2^{2n}, 2^{2n + 1})$に含まれる奇数の元は$C$の元である
            \item $[2^{2n}, 2^{2n + 1})$に含まれる偶数の元は$C$の元である
            \item それ以外の元は$C$に含まれない
        \end{itemize}
        とする。この時、$C = \{1,2,5,7,8,10,\ldots\}$と構成されていくが、
        これは明らかに$a_n(C)\to\frac{1}{2}\ (n\to\infty)$となる(証明は前述とほぼ同じ)。
            
        このとき、$F := 2\NN\cap C\not\in \mathcal{C}$であることを示す。
        $F$の定義を書き直すと、
        \[
            F = \{k\in2\NN;k\in[2^{2n}, 2^{2n+1}),\exists n\in\NN\}
        \]
        である。この$a_n(F)$が収束しないことを示す、すなわち振動することをしめせばいいので、
        適当な$2$つの部分列をとってくればよい。これは、$a_{2^{2n}-1}(F)$と$a_{2^{2n + 1}-1}(F)$を考えればうまくいく。
        \begin{itemize}
            \item $a_{2^{2n} - 1}(F)$の極限
                \begin{align*}
                    a_{2^{2n} - 1}(F) &= \frac{2 + 2^3 + \cdots + 2^{2n - 1}}{2^{2n} - 1} \\
                    &= \frac{2(1 + 2^2 + \cdots + 2^{2n-2})}{2^{2n} - 1} \\
                    &= \frac{2^{2n} - 1}{2^2-1} \cdot \frac{1}{2^{2n} - 1} \\
                    &= \frac{2}{3}
                \end{align*}
            \item $a_{2^{2n + 1} - 1}(F)$の極限
                \begin{align*}
                    a_{2^{2n + 1} - 1}(F) &= 1 - \frac{1 + 2^2 + \cdots + 2^{2n}}{2^{2n + 1} - 1} \\
                    &= 1 - \frac{2^{2n + 2} - 1}{2^2-1} \cdot \frac{1}{2^{2n + 1} - 1} \\
                    &= 1 - \frac{4^{n + 1} - 1}{3(2 \cdot 4^n - 1)} \\
                    &= 1 - \frac{4 - 1/4^n}{6 - 3/4^n} \to 1 - \frac{2}{3} = \frac{1}{3}\ \ (n\to\infty)
                \end{align*}
        \end{itemize}
        以上より二つの部分列をとってきたとき、その極限値が一致しないので$a_n(F)$は振動する、
        すなわちこのExcersizeの主張が正しいことがわかった。

    \subsection{17ページ Borel $\sigma$-algebraの証明について}
        $\BB := \BB(R), \pi(R) :=\{(-\infty, x];x\in\RR\}$とするとき、
        \[
            \BB = \sigma(\pi(\RR))
        \]
        であることを示すのが、Examplesの目的である。
        示すべきことは以下のとおりである。
        \begin{enumerate}[font = \bfseries, label = step \arabic*.]
            \item $I\in\pi(\RR)\Rightarrow I\in \BB$を示す。
            \item $O\in\mathcal{O} \Rightarrow O \in \sigma(\pi(R))$を示す。
            \begin{enumerate}
                \item $O$が可算個の開区間の和で表せる、すなわち、
                    $\exists a_n, b_n\in\RR, a_n < b_n\ (n\in\NN)$があって、
                    \[
                        O = \bigcup_{n\in \NN}(a_n, b_n)
                    \]
                \item このとき、任意の開区間
                    $(a, b)\ (a, b\in \RR)$が
                    $I\in\sigma(\pi(\RR))$の元であることを示す。
            \end{enumerate}
        \end{enumerate}
        \begin{proof}
            $\RR$の開集合全体の集合を$\mathcal{O}$とする。
            \begin{enumerate}[font = \bfseries, label = step \arabic*.]
                \item "$I\in\pi(\RR)\Rightarrow I\in \BB$"\par
                    $I\in\pi(\RR)$とすると、$\exists x\in\RR$があって、
                    \[
                        I = (-\infty, x]
                    \]
                    と表される。これが、可算個の開集合で表されたら良い。このとき、
                    \[
                        I = \bigcap_{n} (-\infty, x + n^{-1})
                    \]
                    を示せば良い(自明として扱っていいが、ここではあえて証明する)。
                    \begin{enumerate}
                        \item "$I\subseteq \bigcap_{n} (-\infty, x + n^{-1})$"\par
                            これは自明である。
                        \item "$\bigcap_{n} (-\infty, x + n^{-1})\subseteq I$"\par
                            $a\in\bigcap_{n} (-\infty, x + n^{-1})$とすると、$\forall n\in\NN$
                            \[
                                -\infty<a<x + n^{-1}
                            \]
                            となる。
                            我々が示したいことは、示したいことは$a\leq x$を満たすことである。
                            これは
                            \[
                                a > x \Rightarrow \exists n \in \NN;\ a > x + n^{-1}
                            \]
                            を示すことによって達成されるが、
                            $a-x > n^{-1} > 0, 1 > 0$
                            なのでアルキメデスの公理
                            \footnote{
                                $a, b > 0\ (a, b\in \RR)$に対してある$n\in\NN$があって
                                \[
                                    an > b
                                \]
                                とすることができる。
                            }
                            より、$\exists n \in \NN$があって、
                            \begin{align*}
                                (a-x)\cdot n &> 1\\
                                a &> x + n^{-1}
                            \end{align*}
                            よって、示した命題のたいううをとって
                            \[
                                x \leq a
                            \]
                            である。
                    \end{enumerate}
                    これにより、$I = \bigcap_{n} (-\infty, x + n^{-1})$を示すことができた。
                    ここで、$I = \bigcap_{n} (-\infty, x + n^{-1}) \in \BB$なので、$I\in\BB$。
    
                    $\therefore$\ 任意の$n\in\NN$について、
                    $(-\infty, x + n^{-1})\in\mathcal{O}$
                    である。これはすなわち、
                    $(-\infty, x + n^{-1})\in\BB$
                    に他ならず、$\BB$は$\sigma$-algebraなので、開区間の可算個の和集合はもちろん$\BB$の元となる。
    
                \item "$O\in\mathcal{O} \Rightarrow O \in \sigma(\pi(R))$"
                    まず、$\RR$が第二可算公理を満たすこと示す。すなわち、
                    $\forall O \in \mathcal{O}$がある可算個の開区間$(a_n, b_n)\ (n\in\NN, a_n \leq b_n)$を用いて、
                    \[
                        O = \bigcup_{n} (a_n, b_n)
                    \]
                    $O$は開集合なので$\forall x \in O,\exists\epsilon > 0;$
                    \[
                        B(x;\epsilon) \subseteq O \Rightarrow (x-\epsilon, x + \epsilon)\subseteq O
                    \]
                    また、$\QQ$は、$\RR$において稠密であるので、$\forall x \in O$について、$\exists\overline{q_x}, \underline{q_x}\in \QQ$があって、
                    \[
                        x - \epsilon < \underline{q_x} < x < \overline{q_x} < x + \epsilon
                    \]
                    とすることができる。このとき、
                    $x \in (\underline{q_x}, \overline{q_x}) \subseteq B(x;\epsilon) \subseteq O$
                    であるので、$Q_x := (\underline{q_x}, \overline{q_x})$と定義すると、
                    \[
                        \bigcup_{x\in O} Q_x \subseteq O
                    \]
                    となる。よって、$O \subseteq \bigcup_{x\in O} Q_x$を示せば良いが、これは自明である。
                    すなわち、
                    \[
                        O = \bigcup_{x\in O} Q_x
                    \]
                    ところで、
                    $\mathcal{I} := \{Q_x;x\in O\}, \mathcal{I}_\QQ := \{(p,q); p,q\in\QQ\}$
                    とすると、明らかに$\mathcal{I} \subseteq \mathcal{I}_\QQ$である。
                    $\mathcal{I}_\QQ$は可算濃度なので、$\mathcal{I}$も可算濃度である。
                    つまり、$O$は$\mathcal{I}$の元の和集合、すなわち、可算個の開区間の和で表せる。
                    
                    よって、任意の$O\in\mathcal{O}$に対して、ある開区間列$\{(a_n, b_n)\}_{n\in\NN}$があって、
                    \[
                        O = \bigcap_{n\in\NN} (a_n, b_n)
                    \]
                    となる。このとき、$(a_n, b_n)\in\BB\ (n\in\NN)$であり、$\BB$は$\sigma$-algebraなので、
                    \[
                        \forall n \in \NN; (a_n, b_n)\in\BB\Rightarrow \bigcup_{n\in\NN}(a_n, b_n)\in\BB
                    \]
                    である。したがって示すことは、"任意の開区間$(a, b)\ (a, b\in \RR)$は$\sigma(\pi(\RR))$の元"である。
                    これを示すためには、
                    \[
                        (a, b) = \bigcup_{n\in\NN}(a, b - \frac{b - a}{n})
                    \]
                    であることを示せばいい。
                    \begin{enumerate}
                        \item "$\bigcup_{n\in\NN}(a, b - \frac{b - a}{n}) \subseteq (a, b)$"\par
                            これは自明である。
                        \item "$(a, b) \subseteq \bigcup_{n\in\NN}(a, b - \frac{b - a}{n})$"\par
                            $x \in (a, b)$とすると、$a < x < b$であるが、このとき、
                            $\exists n\in \NN$があって、
                            \[
                                x < \frac{b-a}{n}
                            \]
                            を示せばいい。$a = b$のときは空集合となり$\sigma(\pi(\RR))$の元であるので、
                            $a < b$と仮定する。すると
                            $b - a > 0, b - x > 0$
                            となるのでアルキメデスの公理より、
                            $\exists n \in \NN$;
                            \begin{align*}
                                (b-a) &< (b-x)n \\
                                \frac{b-a}{n} &< b-x \\
                                x - b &< -\frac{b-a}{n} \\
                                x &< b - \frac{b-a}{n}
                            \end{align*}
                            $\therefore\ x\in (a, b - \frac{b-a}{n})\ (\exists n\in \NN)$
                            すなわち
                            \[
                                (a, b) \subseteq \bigcup_{n\in\NN}(a, b - \frac{b - a}{n})
                            \]
                            が成立する。
                    \end{enumerate}
            \end{enumerate}
        \end{proof}
    