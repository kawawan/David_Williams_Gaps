\section{Chapter 11}
  ここは本当に素晴らしい章。
  しかし若干直観的に示している部分もあるのでそこだけ補足。
  \subsection{11.2}
    \subsubsection{108ページ式(D)について}
      直観的に明らかですが、きちっとした証明をすると少し煩雑でした。
      $\leq$の記号を用いているのは意図的で、$=$が成立するときがちゃんとあります。
      \begin{proof}
        次のように場合分けする。
        \begin{enumerate}
          \item 上向き横断がなかた場合
            \begin{enumerate}
              \item ゲームに参加しなかった場合
              \item ゲームに参加した場合
            \end{enumerate}
          \item 上向き横断があった場合
          \begin{enumerate}
            \item 最後の横断ののち、ゲームに参加しなかった場合
            \item 最後の横断ののち、ゲームに参加した場合
          \end{enumerate}
        \end{enumerate}

        [1.(a)について]
        1.は(a),(b)ともに、$U_N[a,b](\omega) = 0$のときである。
        1.(a)の時は、
        \[Y_N = 0\]
        であるので、
        \[Y_N \ge -(X_N - a)^-\]
        が成立。また、\underline{このときのみ等号が成立}する。

        [1.(b)について]
        ゲームに参加はしているので、どこかの時刻で$X$は$a$を下回っている。
        $m$をその最小の時刻とする。このとき、
        \[
          Y_N = X_N - X_m
        \]
        となるので、
        \[X_N - X_m \ge -(X_N - a)^-\]
        を示せばよい。
        \begin{align*}
          \text{(左辺)}-\text{(右辺)} &= X_N - X_m + (X_N - a)^-\\
          &= \begin{cases}
            X_N - X_m + 0 & (0 \ge -X_N + a)\\
            X_N - X_m - X_N + a & (-X_N + a > 0)
        \end{cases}\\
        &= \begin{cases}
          X_N - X_m  & (X_N \ge a) \\
          a - X_m  & (a > X_N)
        \end{cases}
        \end{align*}
        どちらの場合も、$>0$となる\footnote{$X_m$の定義!}。よって成立。

        [2.(a)の場合]
        $U_N[a,b](\omega) = k$とおく。これは2.(b)でも同様とする。
        上向き横断の定義より、ある時刻$t_k (\leq N)$が存在して、
        $X_{t_k} > b$である。
        横断を一回終えると所持金は少なくとも$(b-a)$は増えるので、
        \[Y_{t_k} > k(b-a) = (b-a) \times U_N[a,b](\omega)\]
        が成立する。
        \begin{align*}
          Y_N &= Y_{t_k} + (Y_N - Y_{t_k}) \\
          &> (b - a)U_N[a,b] + (Y_N - Y_{t_k})
        \end{align*}
        であることに注意すると、あとは
        \[Y_N - Y_{t_k} \ge -(X_N - a)^-\]
        を示せばよい。
        (a)の場合、つまり$t_k + 1$行こうゲームに参加しなかった場合、
        \[Y_N = Y_{t_k}\]
        だから、
        \[Y_N - Y_{t_k} = 0 \ge -(X_N - a)^-\]
        なので、成立する。

        [2.(b)の場合]
        \[Y_N - Y_{t_k} \ge -(X_N - a)^-\]
        を示ばよいというところまでは同じ。
        時刻$(t_k \leq)\, l \,(< N)$から再びゲームに参加したとする。
        このとき、
        \[
          Y_N - Y_{t_k} = Y_{t_k} + X_N - X_l - Y_{t_k} = X_N - X_l
        \]
        なので、
        \begin{equation*}
          X_N - X_l + (X_N - a)^- = \begin{cases}
            X_N - X_l & (X_N \ge a)\\
            a -X_l & (a> X_N)
        \end{cases}
        \end{equation*}
        どちらの場合も$>0$となる。
      \end{proof}

  \subsection{11.5}
    \subsubsection{109ページ12行目}
      \[
        \Lambda_{a,b} \subseteq \{\omega \mid U_{\infty}[a,b](\omega) = \infty\}
      \]
      について。

      \begin{proof}
        $\omega \in \Lambda_{ab}$をとる。
        $U_{\infty}[a,b](\omega) = k < \infty$と仮定する。
        $\Lambda_{ab}$の定義より、
        \[\liminf X_n(\omega) = \sup_{n}\inf_{\nu \ge n}X_{\nu}(\omega) < a\]
        この式から明らかなように
        \[\inf_{\nu \ge n}X_{\nu}(\omega) < a \quad (\forall n)\]
        同様に、
        \[\sup_{\nu \ge n}X_{\nu}(\omega) > b \quad (\forall n)\]
        $t_k$を最後の上向き横断時刻が達成された時刻とする。
        時刻$t_k + 1$について次が成立する。
        \[\inf_{\nu \ge t_k + 1}X_{\nu}(\omega) < a\]
        全ての$\nu \ge t_k + 1$について$X_{\nu} \ge a$とすると
        \footnote{$\min$でとっているならこの仮定はいらないがいまは$\inf$でとっているので必要}、
        $a$は数列$\{X_{\nu}(\omega)\}_{\nu = t_k +1}^{\infty}$の下界である。
        したがって、下限は下界の最大元であるから
        \[\inf_{\nu \ge t_k + 1}X_{\nu}(\omega) \ge a\]
        が従い、矛盾する。
        よって、ある$\xi_1 (\ge t_k + 1)$について
        \[X_{\xi_1} < a\]
        となる。
        同様の議論をすると、ある$\xi_2 (\ge \xi_1)$について
        \[X_{\xi_2} > b\]
        となる。これは、$k+1$回目の上向き横断が実現されたことになる。
        よって矛盾。よって$U_{\infty}[a,b](\omega) = \infty$
        が従う。
      \end{proof}
      \begin{rem*}
        $\limsup$と$\liminf$が一致していないということは、
        絵を描けば$X$のsample pathは上下に振動しまくっていることになる。
        だから直観的には明らかでしょう。
        もう少し短い証明があるかもですね。
      \end{rem*}

\section{Chapter 12}
  \subsection{12.1}
    \subsubsection{110ページ下から2行目}
      9.5節を参照せよとあります。参照しましょう。
      \begin{itembox}[l]{9.5の内容}
        $\mathcal{F} \supseteq \mathcal{G}$で、$\mathcal{G}$は
        $\mathcal{F}$のsub $\sigma$-alg.とする。
        $\mathcal{L}^2(\mathcal{F})$において
        $X$の$\mathcal{G}$に関する条件付き期待値の
        存在を言うとき、$X$に直交射影の定理(6.11参照)を用いて、
        とある確率変数$Y$の存在を言っていた。
        この時の$Y$は次を満たす。
        \begin{itemize}
          \item $(X - Y) \perp Z , \quad \forall Z \in \mathcal{L}^2(\mathcal{G})$
          \item $Y$は$\EE[X \mid \mathcal{G}]$の一変形
        \end{itemize}
      \end{itembox}
      さて、戻りましょう。
      教科書の110ページの下から3行目で
      \[
        \EE[M_v \mid \mathcal{F}_u] = M_u \quad (a.s)
      \]
      を示していた。
