
\subsection{p16 Exercise}
    $C_1\cap C_2\not\in\mathcal{C}$となる$C_1, C_2\in\mathcal{C}$の具体例を示す。
    $a_n(C) := \frac{\#\{k\in C; k\leq n\}}{n}$
    とする。このとき、$2\NN\in\mathcal{C}$であることを示す。
    \begin{align*}
        a_n(2\NN) = \left \{
        \begin{array}{cc}
            \frac{1}{2} & n\in 2\NN\\
            \frac{(n-1)}{2 n} & \mathrm{else}
        \end{array}
        \right.
    \end{align*}
    となるので、
    \begin{align*}
        \sup_{n\leq k} a_k(2\NN) &= \frac{1}{2} \\
        \inf_{n\leq k} a_k(2\NN) &= \frac{n-1}{2n}
    \end{align*}
    で、それぞれの極限は
    \begin{align*}
        \lim_{n\to\infty}\sup_{n\leq k} a_k(2\NN) &= \frac{1}{2} \\
        \lim_{n\to\infty}\inf_{n\leq k} a_k(2\NN) &= \lim_{n\to\infty}\frac{n-1}{2n} = \frac{1}{2}
    \end{align*}
    すなわち、
    $\lim_{n\to\infty} a_n(2\NN) = \frac{1}{2}$
    となり収束するので、
    $2\NN\in\mathcal{C}$。

    次に、$C$を次のように構成する。$n = 0,1,2,\ldots$について、
    \begin{itemize}
        \item $[2^{2n}, 2^{2n + 1})$に含まれる奇数の元は$C$の元である
        \item $[2^{2n}, 2^{2n + 1})$に含まれる偶数の元は$C$の元である
        \item それ以外の元は$C$に含まれない
    \end{itemize}
    とする。この時、$C = \{1,2,5,7,8,10,\ldots\}$と構成されていくが、
    これは明らかに$a_n(C)\to\frac{1}{2}\ (n\to\infty)$となる(証明は前述とほぼ同じ)。
        
    このとき、$F := 2\NN\cap C\not\in \mathcal{C}$であることを示す。
    $F$の定義を書き直すと、
    \[
        F = \{k\in2\NN;k\in[2^{2n}, 2^{2n+1}),\exists n\in\NN\}
    \]
    である。この$a_n(F)$が収束しないことを示す、すなわち振動することをしめせばいいので、
    適当な$2$つの部分列をとってくればよい。これは、$a_{2^{2n}-1}(F)$と$a_{2^{2n + 1}-1}(F)$を考えればうまくいく。
    \begin{itemize}
        \item $a_{2^{2n} - 1}(F)$の極限
            \begin{align*}
                a_{2^{2n} - 1}(F) &= \frac{2 + 2^3 + \cdots + 2^{2n - 1}}{2^{2n} - 1} \\
                &= \frac{2(1 + 2^2 + \cdots + 2^{2n-2})}{2^{2n} - 1} \\
                &= \frac{2^{2n} - 1}{2^2-1} \cdot \frac{1}{2^{2n} - 1} \\
                &= \frac{2}{3}
            \end{align*}
        \item $a_{2^{2n + 1} - 1}(F)$の極限
            \begin{align*}
                a_{2^{2n + 1} - 1}(F) &= 1 - \frac{1 + 2^2 + \cdots + 2^{2n}}{2^{2n + 1} - 1} \\
                &= 1 - \frac{2^{2n + 2} - 1}{2^2-1} \cdot \frac{1}{2^{2n + 1} - 1} \\
                &= 1 - \frac{4^{n + 1} - 1}{3(2 \cdot 4^n - 1)} \\
                &= 1 - \frac{4 - 1/4^n}{6 - 3/4^n} \to 1 - \frac{2}{3} = \frac{1}{3}\ \ (n\to\infty)
            \end{align*}
    \end{itemize}
    以上より二つの部分列をとってきたとき、その極限値が一致しないので$a_n(F)$は振動する、
    すなわちこのExcersizeの主張が正しいことがわかった。
