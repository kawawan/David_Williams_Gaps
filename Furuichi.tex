\section{Chapter 4}
\subsection{Independence p.40}
 
\begin{def*}
$\mathcal{F}$の部分$\sigma$-加法族$\mathcal{G}_{1}, \mathcal{G}_{2}, \ldots$が独立であるとは,$G_{i}\in \mathcal{G}_{i}(i\in \mathbb{N})$で$i_{1}, \ldots i_{n}$が相異なるとき,つねに
$$
\mathbb{P}\left(G_{i_{1}}\cap \cdots \cap G_{i_{n}}\right)
=\prod_{k=1}^{n}\mathbb{P}\left(G_{i_{k}}\right)
$$
が成り立つこと.
\end{def*}

\begin{def*}
 確率変数$X_{1}, X_{2}, \ldots$が独立であるとは, $\sigma$-加法族$\sigma(X_{1}), \sigma(X_{2}), \ldots$が独立であること.
\end{def*}

\begin{def*}
事象$E_{1}, E_{2}, \ldots$が独立であるとは, $\sigma$-加法族$\varepsilon_{1}, \varepsilon_{2}, \ldots$が独立であること.
ただしここで, $\varepsilon_{n}$は$\sigma$-加法族$\{\phi, E_n, \Omega \setminus E_{n}, \Omega\}$である.
\end{def*}
 
\begin{lem*}

\end{lem*}
