% Chapter11から
\section{Chapter 11}
  ここは本当に素晴らしい章。
  しかし若干直観的に示している部分もあるのでそこだけ補足。
  \subsection{11.2}
    \subsubsection{108ページ式(D)について}
      直観的に明らかですが、きちっとした証明をすると少し煩雑でした。
      $\leq$の記号を用いているのは意図的で、$=$が成立するときがちゃんとあります。
      \begin{proof}
        次のように場合分けする。
        \begin{enumerate}
          \item 上向き横断がなかった場合
            \begin{enumerate}
              \item ゲームに参加しなかった場合
              \item ゲームに参加した場合
            \end{enumerate}
          \item 上向き横断があった場合
          \begin{enumerate}
            \item 最後の横断ののち、ゲームに参加しなかった場合
            \item 最後の横断ののち、ゲームに参加した場合
          \end{enumerate}
        \end{enumerate}

        [1.(a)について]
        1.は(a),(b)ともに、$U_N[a,b](\omega) = 0$のときである。
        1.(a)の時は、
        \[Y_N = 0\]
        であるので、
        \[Y_N \ge -(X_N - a)^-\]
        が成立。また、\underline{このときのみ等号が成立}する。

        [1.(b)について]
        ゲームに参加はしているので、どこかの時刻で$X$は$a$を下回っている。
        $m$をその最小の時刻とする。このとき、
        \[
          Y_N = X_N - X_m
        \]
        となるので、
        \[X_N - X_m \ge -(X_N - a)^-\]
        を示せばよい。
        \begin{align*}
          \text{(左辺)}-\text{(右辺)} &= X_N - X_m + (X_N - a)^-\\
          &= \begin{cases}
            X_N - X_m + 0 & (0 \ge -X_N + a)\\
            X_N - X_m - X_N + a & (-X_N + a > 0)
        \end{cases}\\
        &= \begin{cases}
          X_N - X_m  & (X_N \ge a) \\
          a - X_m  & (a > X_N)
        \end{cases}
        \end{align*}
        どちらの場合も、$>0$となる\footnote{$X_m$の定義!}。よって成立。

        [2.(a)の場合]
        $U_N[a,b](\omega) = k$とおく。これは2.(b)でも同様とする。
        上向き横断の定義より、ある時刻$t_k (\leq N)$が存在して、
        $X_{t_k} > b$である。
        横断を一回終えると所持金は少なくとも$(b-a)$は増えるので、
        \[Y_{t_k} > k(b-a) = (b-a) \times U_N[a,b](\omega)\]
        が成立する。
        \begin{align*}
          Y_N &= Y_{t_k} + (Y_N - Y_{t_k}) \\
          &> (b - a)U_N[a,b] + (Y_N - Y_{t_k})
        \end{align*}
        であることに注意すると、あとは
        \[Y_N - Y_{t_k} \ge -(X_N - a)^-\]
        を示せばよい。
        (a)の場合、つまり$t_k + 1$行こうゲームに参加しなかった場合、
        \[Y_N = Y_{t_k}\]
        だから、
        \[Y_N - Y_{t_k} = 0 \ge -(X_N - a)^-\]
        なので、成立する。

        [2.(b)の場合]
        \[Y_N - Y_{t_k} \ge -(X_N - a)^-\]
        を示ばよいというところまでは同じ。
        時刻$(t_k \leq)\, l \,(< N)$から再びゲームに参加したとする。
        このとき、
        \[
          Y_N - Y_{t_k} = Y_{t_k} + X_N - X_l - Y_{t_k} = X_N - X_l
        \]
        なので、
        \begin{equation*}
          X_N - X_l + (X_N - a)^- = \begin{cases}
            X_N - X_l & (X_N \ge a)\\
            a -X_l & (a> X_N)
        \end{cases}
        \end{equation*}
        どちらの場合も$>0$となる。
      \end{proof}

  \subsection{11.5}
    \subsubsection{109ページ12行目}
      \[
        \Lambda_{a,b} \subseteq \{\omega \mid U_{\infty}[a,b](\omega) = \infty\}
      \]
      について。

      \begin{proof}
        $\omega \in \Lambda_{ab}$をとる。
        $U_{\infty}[a,b](\omega) = k < \infty$と仮定する。
        $\Lambda_{ab}$の定義より、
        \[\liminf X_n(\omega) = \sup_{n}\inf_{\nu \ge n}X_{\nu}(\omega) < a\]
        この式から明らかなように
        \[\inf_{\nu \ge n}X_{\nu}(\omega) < a \quad (\forall n)\]
        同様に、
        \[\sup_{\nu \ge n}X_{\nu}(\omega) > b \quad (\forall n)\]
        $t_k$を最後の上向き横断時刻が達成された時刻とする。
        時刻$t_k + 1$について次が成立する。
        \[\inf_{\nu \ge t_k + 1}X_{\nu}(\omega) < a\]
        全ての$\nu \ge t_k + 1$について$X_{\nu} \ge a$とすると
        \footnote{$\min$でとっているならこの仮定はいらないがいまは$\inf$でとっているので必要}、
        $a$は数列$\{X_{\nu}(\omega)\}_{\nu = t_k +1}^{\infty}$の下界である。
        したがって、下限は下界の最大元であるから
        \[\inf_{\nu \ge t_k + 1}X_{\nu}(\omega) \ge a\]
        が従い、矛盾する。
        よって、ある$\xi_1 (\ge t_k + 1)$について
        \[X_{\xi_1} < a\]
        となる。
        同様の議論をすると、ある$\xi_2 (\ge \xi_1)$について
        \[X_{\xi_2} > b\]
        となる。これは、$k+1$回目の上向き横断が実現されたことになる。
        よって矛盾。よって$U_{\infty}[a,b](\omega) = \infty$
        が従う。
      \end{proof}
      \begin{rem*}
        $\limsup$と$\liminf$が一致していないということは、
        絵を描けば$X$のsample pathは上下に振動しまくっていることになる。
        だから直観的には明らかでしょう。
        もう少し短い証明があるかもですね。
      \end{rem*}

\section{Chapter 12}
  \subsection{12.1}
    \subsubsection{110ページ下から2行目}
      9.5節を参照せよとあります。参照しましょう。
      \begin{itembox}[l]{9.5の内容}
        $\mathcal{F} \supseteq \mathcal{G}$で、$\mathcal{G}$は
        $\mathcal{F}$のsub $\sigma$-alg.とする。
        $\mathcal{L}^2(\mathcal{F})$において
        $X$の$\mathcal{G}$に関する条件付き期待値の
        存在を言うとき、$X$に直交射影の定理(6.11参照)を用いて、
        とある確率変数$Y$の存在を言っていた。
        この時の$Y$は次を満たす。
        \begin{itemize}
          \item $(X - Y) \perp Z , \quad \forall Z \in \mathcal{L}^2(\mathcal{G})$
          \item $Y$は$\EE[X \mid \mathcal{G}]$の一変形
        \end{itemize}
      \end{itembox}
      さて、戻りましょう。
      教科書の110ページの下から3行目で
      \[
        \EE[M_v \mid \mathcal{F}_u] = M_u \quad (a.s)
      \]
      を示していた。$\mathcal{F}_v$は完備より、6.11の定理を使うと、
      ある$Y \in \mathcal{L}^2(\mathcal{F}_u)$が存在して
      \[(M_v - Y) \perp Z \quad (\forall Z \in \mathcal{L}^2(\mathcal{F}_u))\]
      となる。
      また、9.5よりこの$Y$は$\EE[M_v \mid \mathcal{F}_u]$の一変形だった。
      条件付き期待値はalmost surely で一意的なので、
      \[
        Y = M_u \quad (a.s)
      \]
      したがって、
      \[
        (M_v - M_u) \perp Z \quad (\forall Z \in \mathcal{L}^2(\mathcal{F}_u))
      \]
      となり、$(M-v -M_u)$は$\mathcal{L}^2(\mathcal{F}_u)$に直交する。
      $M_t, M_s$はともに$\mathcal{F}_u$可測で、
      $M$が$\mathcal{L}^2$有界より、
      \[
        M_t, \, M_s \in \mathcal{L}^2(\mathcal{F}_u)
      \]
      ゆえに、
      \[
        M_t - M_s \in \mathcal{L}^2(\mathcal{F}_u)
      \]
      したがって、とりわけ110ページの式(a)が成立するのである。

    \subsubsection{111ページTHEOREMの証明}
      直観的には明らかだけど...。

      \begin{proof}
        $(\Rightarrow)$
        $M$は$\mathcal{L}^2$有界とする。
        (b)式の右辺より、数列$\{\EE[M_n^2]\}_{n = 0}^{\infty}$
        は単調増加な数列で、仮定より、上に有界である。
        したがって、極限
        \[
          \lim_{n \to \infty}\EE[M_n^2]
        \]
        は有限の値に収束する。
        すなわち、
        \[
          \EE[M_0^2] + \sum_k \EE[(M_k - M_{k-1})^2)] < \infty
        \]
        したがって(c)の式が成立する。

        $(\Leftarrow)$
        (c)が成立するとする。
        \[
          \EE[M_n^2] \leq \EE[M_0^2] + \sum_k \EE[(M_k - M_{k-1})^2] < \infty
        \]
        である。
        $\EE[M_0^2] + \sum_k \EE[(M_k - M_{k-1})^2]$は
        $\{\EE[M_n^2]\}_{n = 0}^{\infty}$の上界である。
        よって、
        \[
          \sup_{n} \EE[M_n^2] \leq \EE[M_0^2] + \sum_k \EE[(M_k - M_{k-1})^2] < \infty
        \]
      \end{proof}

    \subsubsection{111ページ(c)式の2行下}
      $M_{\infty}$が$\mathcal{L}^2$に存在するのはFatou Lemma か言えたり。
      \begin{align*}
        \EE[M_{\infty}^2] = \EE[\liminf M_n^2] &\leq \liminf\EE[M_n^2]\\
        &\leq \sup_n \EE[M_n^2]\\
        &< \infty
      \end{align*}

    \subsubsection{111ページ一番下の行}
      (e)式が$=$でも成立するのは、
      \[
        \lim_{r \to \infty} \EE[(M_{n+r} - M_n)^2] = \EE[(M_{\infty} - M_n)^2]
      \]
      が成立するから。なおこの式が成立するのはノルムは連続関数だから。あ

  \subsection{112ページNotationの$A_n$の定義について}
    $A_n$は分散の和なので、実数列。確率変数列じゃないよ!
  \subsection{112ページ(*)の式の二つ目の$=$}
    分散を計算。
  \subsection{113ページ上から8行目}
    $N$がmaritingaleになることは計算しましょう。
    ここまでの議論で
    \[\sigma_k^2 = \EE[M_k^2 \mid \mathcal{F}_{k-1}] - M_{k-1}^2\]
    がalmost surelyで成り立つことが分かっている。
    この等式を用いて、
    \[\EE[N_m \mid \mathcal{F}_{m-1}] = N_{m-1}\]
    がalmost surelyで成り立つことを証明しよう。
  \subsection{113ページ上から12行目}
    $N^T$はmartingaleだからTower Propertyを用いれば、
    \[\EE[N_n^T \mid \mathcal{F}_0] = N_0 = 0\]
    が分かる。
  \subsection{113ページ(b)の証明の最後の三行}
    ここは不親切。"bounded"の意味を取り違えないように。
    翻訳すれば、
    「しかしながら、$\sum X_n$がa.s.で収束するから、任意の部分和$\sum_{k=1}^n X_k$
    も上に有界である。したがって、うまく実数$c$をとれば、$\PP(T= \infty) > 0$
    とすることができる。このことと、(**)の式により、
    $A_{\infty} := \sum \sigma_k^2 < \infty$であることが分かる。」
    のようになる。
    (この部分の日本語訳の本を見てみましたが、あまり訳がうまくないとおもいます。
    逐語訳を意識しすぎているのと、学術的な言いまわしに凝りすぎていて日本語として意味が
    掴みにくいです。)

    さて、「部分和が上に有界」とは、次のような意味です。
    \begin{center}
      $\omega$を一つ固定するごとに、$n$に関する数列
      $\{\sum_{k=1}^n X_k(\omega)\}$は上に有界である。
    \end{center}
    次のような意味ではないことに気を付けてください\footnote{このような誤解を生んでしまうような書き方は避けるべきだと思います。
    丁寧に「$\omega$を任意に一つ固定すれば」という文言を付け足せば、
    それだけで悲しい誤解を生まなくて済むのに。}。
    \begin{center}
      関数列$\{\sum_{k=1}^n X_k\}_{n \in \NN}$は一様有界である。すなわち、
      \[\exists K ,\forall \omega, \forall n, \quad |\sum_{k=1}^n X_k(\omega)| \leq K\]
    \end{center}

    \subsubsection{$\PP(T= \infty) > 0$について}
      $Y$を次のように置きます。
      \[Y(\omega) := \sup_n \left|\sum_{k=1}^n X_k(\omega) \right|\]
      先ほど言ったように、$Y(\omega) < \infty$です。
      よって、$Y(\omega) < c$となるような実数$c$があります。
      \[\PP(Y < c) \rightarrow 1 \quad (c \rightarrow \infty)\]
      測度の単調性から$c$を大きくしていけばいつしか$\PP(Y < c)$は$0$を超えます。
      だからそのような$c$をとれば、$T_c := \inf\{r \mid |M_r| > c\}$について
      \[\PP(T_c = \infty) > 0\]
      となります。

    \subsubsection{証明の完成}
      前小小節でとった$\PP(T_c = \infty) > 0$となるような$c$について次のような式が成り立ちます。
      \begin{align*}
        \EE[A_{T \wedge n}] \leq& (K + c)^2 \\
        \EE[\sum_{k = 1}^{T_c \wedge n}\Var(X_k)] =& \left( \sum_{k=1}^n \Var(X_k)\right) \PP(T_c > n)\\
        &+ \left( \sum_{k=1}^{T_c} \Var(X_k)\right) \PP(T_c \leq n)\\
        (K + c)^2 \ge& \left( \sum_{k=1}^n \Var(X_k)\right) \PP(T_c > n)\\
        (K + c)^2 \ge& \left( \sum_{k=1}^{\infty} \Var(X_k)\right) \PP(T_c = \infty)\\
        \sum_{k=1}^{\infty} \Var(X_k) \leq& \frac{(K + c)^2}{\PP(T_c = \infty)}
      \end{align*}

  \subsection{114ページ2行目}
    "only if"の方、つまり
    \[\sum \varepsilon_n a_n \quad \text{converges(a.s.)} \Rightarrow \sum a_n^2 < \infty\]
    はここの文章に書かれている以上の条件がいります。
    それは、$\{\varepsilon_n a_n\}_{n \in \NN}$が一様有界であることです。
    つまり数列$\{a_n\}_{n \in \NN}$が有界であるという条件が必要です。

  \subsection{114ページ4行目}
    もう8章も前のことなので、影が薄いですが、「確率変数列の和が各点収束する」
    という事象は末尾$\sigma$-alg.の事象です。
    つまり、この事象が起こる確率はコルモゴロフの0-1法則より、
    1か0です。このことに注意しましょう。

  \subsection{115ページ下から14行目}
    (12.2.a)を適用しているのは$(X_n - \EE[X_n])$に使っています。
    使うからにはこれが独立確率変数の列であることと、
    各$n$について平均が0で分散が有限であることを確かめることを忘れずに。

  \subsection{116ページ証明の5行目}
    "... for all but finitely many $n$"という文章は直観的でパット見分かりやすいですが、
    数式で書いた方がよいと思います。そうでないと扱い方が分かりません。
    (BC1)で従うのは、
    \[
      \PP(\limsup \{|X_n| > K\}) = 0
    \]
    であること。$\limsup\{\text{事象}F\}$は、「事象$F$が無限回起こること」
    を意味している。
    これは、「有限個の$n$を除いて事象$F$が起こる」ことではない\footnote{実質無限回起きてるけど}。
    文学的な表現だけを用いて証明を進めると
    このように意味が分からなくなってしまうので、
    数式を書くべきところではちゃんとするべきだと思いました。
    $i.o.$や$e.v.$という用語も直観的に議論を進めるためには大いに重要だと思います。
    しかし、数式に戻してちゃんと議論をしたいときには
    その数式を証明に明記しておかないと分かりにくいと思います。
    こういうことが多いからこの本嫌いなんだよ。
    \footnote{"for all but finitely mayn $n$"という表現もここが初出じゃないの?BC1を証明した時点で$i.o.$や$e.v.$という用語を導入しているのだから、そっちを使ってほしい。}

    というわけで、どのようにして証明5行目の式が得られたかを書きます。
    (BC1)より、$\PP(\limsup\{|X_n| > K\}) = 0$が得られた。
    この集合の補集合を考える。
    \begin{align*}
      1 &= \PP((\limsup\{|X_n| > K\})^c) \\
      &= \PP(\liminf\{|X_n| \leq K\})) \\
      &= \PP(\liminf\{X_n^K = X_n\})\\
      &= \PP(X_n^K \neq X_n \quad e.v.)\\
      &= \PP\left(\bigcup_m \bigcap_{n \ge m}\{X_n^K = X_n\}\right)
    \end{align*}

  \subsection{116ページ証明の6行目}
    "It is therefore clear that we need only show that ...;"のところ。
    $\omega$を固定するごとに有限個の$n$しか$X_n(\omega) \neq X_n^K(\omega)$
    とならない。だから直観的には$\sum X_n^K$がalmost surely で収束することを
    示せば良さそうです。じつは、
    \[
      \sum X_n \text{が収束}(a.s) \Leftrightarrow \sum X_n^K \text{が収束}(a.s.)
    \]
    です。
    これを示すためのちょっとした補題。
    \begin{lem*}
      $\PP$を確立測度とする。$A,B$は確率$1$の事象とする。
      このとき、$\PP(A \cap B) = 1$である。
    \end{lem*}
    \begin{proof}
      $\PP((A \cap B)^c) = \PP(A^c \cup B^c) = 0$であることが仮定よりわかる。
    \end{proof}
    先ほどの同値命題の証明です。
    \begin{proof}
      \begin{gather*}
        A = \bigcup_m \bigcap_{n \ge m}\{X_n^K = X_n\} \\
        B = \{\sum X_n \text{が収束}\} \\
        C =\{\sum X_n^K \text{が収束}\}
      \end{gather*}
      とおく。

      ($\Rightarrow$)

      $\omega \in A \cap B$をとる
      \footnote{$A \cap B \neq \emptyset$は確認してください}。
      このとき、ある$n_{\omega}$が存在して、すべての$n \ge n_{\omega}$
      にたいして、$X_n^K(\omega) = X_n (\omega)$になる。
      このとき、
      \begin{align*}
        \sum X_n^K(\omega) &= \sum_{k=1}^{n_{\omega}-1} X_k^K(\omega) + \sum_{k\ge n_{\omega}} X_k^K(\omega) \\
        &=\sum_{k=1}^{n_{\omega}-1} X_k^K(\omega) + \sum_{k\ge n_{\omega}} X_k(\omega)
      \end{align*}
      また、
      \[
        \sum X_n(\omega) = \sum_{k=1}^{n_{\omega}-1} X_k(\omega) + \sum_{k\ge n_{\omega}} X_k(\omega)
      \]
      であり、$\omega \in B$であることに気を付ければ$\omega \in C$であることがわかる
      \footnote{ここちょっと面倒くさくなった(おなかすいててん)}。
      よって、$A \cap B \subset C$が分かる。
      よって先ほどの補題より、$\PP(C) = 1$であることが分かる。

      同様にすれば$(\Leftarrow)$もわかる。
    \end{proof}
